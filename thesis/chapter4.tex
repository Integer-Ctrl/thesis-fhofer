\chapter{Evaluation}\label{evaluation}

This chapter focuses on the evaluation of the transfer pipeline. First, the individual steps of the pipeline are analyzed and evaluated to examine intermediate results and identify potential weaknesses, starting with an assessment of the scores assigned to the selected passages from the source datasets. Next, the different approaches for selecting candidate documents from the target document corpus are evaluated to determine the most effective approach.
\\\\
The evaluation of the candidate retrieval approaches can only be conducted by applying the transfer pipeline to a target dataset that already contains query relevance judgments. This is necessary because the goal of the candidate retrieval is to pre-select candidate documents from a target corpus that are likely to be relevant to a query. To measure this accurately, the target documents must be labeled with existing relevance judgments.
\\\\
The second part of the chapter focuses on the evaluation of the inferred relevance judgments through two transfer strategies. The first evaluation applies the transfer pipeline to a source dataset and uses the same dataset as the target. This self-transfer evaluation measures how well the pipeline reproduces known relevance judgments within the same dataset. The second evaluation assesses the final transfer to \texttt{ClueWeb22/b} as the target dataset. To facilitate this evaluation, a pooling process is conducted on the candidate documents selected by the best-performing candidate retrieval approach. This is followed by a manual relevance judgment of the pooled documents. The quality of the transferred relevance judgments to \texttt{ClueWeb22/b} is then evaluated based on these manual relevance judgments.
% ================
% Rank Correlation
% ================
\section{Rank Correlation}\label{rank-correlation}

Before starting the actual evaluation of the transfer pipeline, I present the metrics used to evaluate the relevance judgments produced by the pipeline against the actual relevance judgments provided by the retrieval tasks. This clarification is important, as the evaluation results of the passage scores (Section~\ref{passage-scoring}) and the final relevance judgments (Section~\ref{pairwise-preferences}) generated by the transfer pipeline are based on these metrics.
\\\\
The \texttt{Inter-Annotator Agreement} (\texttt{IAA})~\citep{artstein:2017} is a statistical measure that quantifies the consistency between annotations provided by multiple annotators. It indicates the level of agreement or disagreement among annotators when labeling the same dataset, thereby offering insight into the quality of the annotations. In this thesis, the ground truth annotation is represented by the relevance judgments provided by the retrieval tasks. These judgments (annotations) are compared to the inferred relevance judgments generated by the transfer pipeline. A high \texttt{IAA} score indicates a strong agreement between the actual relevance labels and the inferred relevance labels.
\\\\
Various methods exist for calculating \texttt{IAA}, each with its strengths and weaknesses. In this thesis, the retrieval tasks provide judgments as integer labels, i.e., $\in \mathbb{N}$, while the inferred relevance labels are represented as actual relevance scores, i.e., $\in \mathbb{R}^{+}$. Consequently, not every correlation method is suitable for computing the aggreement between the relevances. Therefore, the correlation metrics used to evaluate the level of agreement are \texttt{Kendall's $\tau$} and \texttt{Spearman's $\rho$}. Both metrics compute the rank correlation between linear orders~\citep{monjardet:1998}, making them well-suited to the nature of the retrieval tasks labels and inferred relevance scores in this thesis.
\\\\
Table~\ref{tab:rank-correlation} presents an example of how the correlation metrics are calculated based on some sample labels of the relevance judgments. The table shows the agreement between example reference scores $[0.2, 0.7, 0.5]$ and three label sets: the first two with an expected correlation of $1$, and the last one with lower correlation. The \texttt{Default} column represents the actual rank correlation scores computed by the standard algorithms of \texttt{Kendall's $\tau$} and \texttt{Spearman's $\rho$}.
\\\\\\\\\\\\\\
In contrast to the default algorithms, the desired rank correlation for this thesis, is characterized by the ability to map the inferred relevance scores to the actual relevance labels from the retrieval task in the best possible way. For example, the relevance scores $[0.2, 0.7, 0.5]$ should be perfectly mapped to the label set $[0, 1, 1]$ in the second row of Table~\ref{tab:rank-correlation}. This could be achieved by assuming that all relevance scores smaller equal $0.2$ correspond to label $0$ and all scores greater $0.2$ correspond to label $1$. Therefore, the intended rank correlation for this example should be $1$, in this thesis. To achieve this, a modified variant called \texttt{Greedy} was tested alongside the default versions. This variant greedily maps the reference scores to the label set, representing the idealized rank correlation outcome. This example demonstrates that the default correlation metrics are sensitive to the rank order of the relevance judgments, whereas the \texttt{Greedy} variant avoids this sensitivity, achieving the maximum possible correlation scores.


\begin{table}[t]
  \centering
  \caption{Comparison of rank correlation metrics computed by Kendall's $\tau$ and Spearman's $\rho$. The table shows the agreement between the reference scores $[0.2,0.7,0.5]$ and three label sets: two with an expected correlation of $1$, and one with lower correlation. Default represents the actual rank correlation scores, while Greedy shows the scores obtained by greedily mapping the reference scores to the label set, representing the idealized rank correlation outcome.}
  \label{tab:rank-correlation}
  \begin{tabular}{ccccc}
      \toprule
      \textbf{Comparative Set} & \multicolumn{2}{c}{\textbf{Default}} & \multicolumn{2}{c}{\textbf{Greedy}} \\
      \cmidrule(lr){2-3} \cmidrule(lr){4-5}
                               & $\tau$ & $\rho$ & $\tau$ & $\rho$ \\
      \midrule
      
      $[0, 2, 1]$ & 1.00 & 0.99  & 1.00  & 1.00 \\
      $[0, 1, 1]$ & 0.82 & 0.92  & 1.00  & 1.00 \\
      $[0, 0, 1]$ & 0.00 & 0.11  & 0.50  & 0.50 \\
      \bottomrule
  \end{tabular}
\end{table}


% ==================
% Document Selection
% ==================
\section{Document Selection}\label{eval-doucment-selection}

The first step in the transfer pipeline is to select and segment a subset of documents for each query of a retrieval task from the source datasets. The selected documents play a crucial role in two subsequent steps of the pipeline: (1) the selection of candidate documents from the target corpus and (2) the pairwise preference comparison using \texttt{DouPrompt}. Since only a subset of documents is necessary for each query, two selection conditions were applied.
\begin{table}[t]
  \centering
  \caption{The table presents the results of the candidate selection process, showing the minimum (min), maximum (max), and average (mean) number of documents per query for each source dataset. A maximum of 50 documents per query is possible.}
  \label{tab:document-selection}
  \resizebox{\textwidth}{!}{%
  \begin{tabular}{lcccccc}
      \toprule
      \textbf{Docs./Query} & \textbf{Touché 20} & \textbf{Robust04} & \textbf{TREC-7} & \textbf{TREC-8} & \textbf{TREC-19 DL} & \textbf{TREC-20 DL}  \\

      \midrule

      Label 0        & $27.9$ & $50.0$ & $50.0$ & $50.0$ & $ 49.6$ & $ 50.0$ \\
      Label 1        & $ 6.0$ & $32.8$ & $40.2$ & $40.5$ & $ 31.1$ & $ 28.5$ \\
      Label 2        & $13.0$ & $ 3.8$ &    -   &    -   & $ 23.8$ & $ 16.0$ \\
      Label 3        &    -   &    -   &    -   &    -   & $ 11.7$ & $ 10.3$ \\
      \midrule
      \textbf{Total} & $46.9$ & $86.6$ & $90.2$ & $90.5$ & $116.2$ & $104.8$ \\

      \bottomrule
  \end{tabular}}
\end{table}
\\\\\\\\
First, only documents with at least one relevance judgment in the \texttt{qrel store} of a retrieval task are considered for selection, as documents without relevance judgments do not have information to transfer. Second, a maximum of 50 relevance judgments per relevance label for each query (referred to as a query-label combination) is selected to limit the number of processed documents. Third, the number of documents for each query-label combination is capped at the minimum number of judged documents across all labels for that query, to prevent bias towards labels with a higher number of relevance judgments.
\\\\
The results of the selection process under these conditions are presented in Table~\ref{tab:document-selection}. An average close to 50 documents per query is aimed for. \texttt{TREC-7} and \texttt{TREC-8} achieve the best results, with averages above 40 documents per query. \texttt{Robust04} remains sufficient with nearly 30 documents per query, while the other three datasets fall short. \texttt{Touché 20} has a maximum of only 12 documents for any query, and although \texttt{TREC-19 DL} and \texttt{TREC-20 DL} have higher maximum values, their averages are also very low.
\\\\
The low averages are likely due to the limited number of relevance judgments per query provided by the retrieval tasks, see Tabel~\ref{tab:datasets}. Additionally, the third condition that caps the number of documents for each query-label combination to the minimum number of judged documents across all labels for that query has a significant impact. For example, if a query has 40 relevant documents but only 5 non-relevant documents, both categories are capped at 5 documents during the selection process. This small number of selected relevance judgments and their associated documents could negatively affect the quality of the candidate selection with the nearest neighbor approaches, and the final inferred relevance judgments.
%\\\\
% The segmentation of the selected documents was performed using the \texttt{trec-cast-tools} GitHub repository\footnote{\url{https://github.com/grill-lab/trec-cast-tools}}, which splits documents into passages of up to 250 words in length. This segmentation process is independent of the transfer pipeline and is therefore not evaluated in this thesis.
\pagebreak

% ===============
% Passage Scoring
% ===============
\section{Passage Scoring}\label{eval-passage-scoring}

\begin{table}[t]
    \centering
    \caption{5-fold cross-validation correlations of the assigned passage scores with the original document judgments, reported using Kendall's $\tau$ and Spearman's $\rho$.}
    \label{tab:passage-scoring}
    \resizebox{\textwidth}{!}{%
    \renewcommand{\arraystretch}{1.2}
    \begin{tabular}{clcccccccccccccc}
        \toprule
        \multicolumn{2}{c}{\textbf{Retrieval Model}} & \multicolumn{2}{c}{\textbf{Touché 20}} & \multicolumn{2}{c}{\textbf{Robust04}} & \multicolumn{2}{c}{\textbf{TREC-7}} & \multicolumn{2}{c}{\textbf{TREC-8}} & \multicolumn{2}{c}{\textbf{TREC-19 DL}} & \multicolumn{2}{c}{\textbf{TREC-20 DL}} & \multicolumn{2}{c}{\textbf{Avg.}} \\
        \cmidrule(lr){3-4} \cmidrule(lr){5-6} \cmidrule(lr){7-8} \cmidrule(lr){9-10} \cmidrule(lr){11-12} \cmidrule(lr){13-14} \cmidrule(lr){15-16}
                                                   & & $\tau$ & $\rho$ & $\tau$ & $\rho$ & $\tau$ & $\rho$ & $\tau$ & $\rho$ & $\tau$ & $\rho$ & $\tau$ & $\rho$ & $\tau$ & $\rho$ \\
        \midrule
        \multirow{10}{*}{\rotatebox{90}{\texttt{ndcg10}}}
            & BM25 & 0.802 & 0.871 & 0.0 & 0.0 & 0.801 & 0.896 & 0.813 & 0.906 & 0.725 & 0.831 & 0.755 & 0.846 & 0.649 & 0.725 \\
            & DFR\_BM25 & 0.804 & 0.874 & 0.0 & 0.0 & 0.8 & 0.896 & 0.813 & 0.906 & 0.727 & 0.833 & 0.755 & 0.846 & 0.65 & 0.726 \\
            & DFIZ & 0.778 & 0.852 & 0.0 & 0.0 & 0.796 & 0.896 & 0.8 & 0.898 & 0.73 & 0.836 & 0.753 & 0.845 & 0.643 & 0.721 \\
            & DLH & 0.839 & 0.901 & 0.0 & 0.0 & 0.81 & 0.903 & 0.825 & 0.915 & 0.735 & 0.839 & 0.754 & 0.845 & 0.66 & 0.734 \\
            & DPH & 0.783 & 0.856 & 0.0 & 0.0 & 0.799 & 0.895 & 0.809 & 0.903 & 0.725 & 0.831 & 0.753 & 0.844 & 0.645 & 0.721 \\
            & DirichletLM & 0.679 & 0.752 & 0.0 & 0.0 & 0.798 & 0.896 & 0.802 & 0.898 & 0.721 & 0.825 & 0.751 & 0.841 & 0.625 & 0.702 \\
            & Hiemstra\_LM & 0.855 & 0.914 & 0.0 & 0.0 & 0.807 & 0.902 & 0.819 & 0.91 & 0.729 & 0.834 & 0.759 & 0.849 & 0.661 & 0.735 \\
            & LGD & 0.801 & 0.874 & 0.0 & 0.0 & 0.799 & 0.897 & 0.802 & 0.899 & 0.73 & 0.835 & 0.757 & 0.848 & 0.648 & 0.726 \\
            & PL2 & 0.805 & 0.873 & 0.0 & 0.0 & 0.804 & 0.898 & 0.822 & 0.913 & 0.727 & 0.833 & 0.757 & 0.848 & 0.652 & 0.728 \\
            & TF\_IDF & 0.807 & 0.876 & 0.0 & 0.0 & 0.801 & 0.896 & 0.814 & 0.907 & 0.726 & 0.832 & 0.755 & 0.847 & 0.65 & 0.726 \\
        \midrule
        \multirow{10}{*}{\rotatebox{90}{\texttt{p10}}}
            & BM25 & 0.765 & 0.828 & 0.0 & 0.0 & 0.792 & 0.872 & 0.807 & 0.885 & 0.657 & 0.756 & 0.697 & 0.785 & 0.62 & 0.688 \\
            & DFR\_BM25 & 0.768 & 0.833 & 0.0 & 0.0 & 0.791 & 0.87 & 0.807 & 0.884 & 0.658 & 0.757 & 0.696 & 0.785 & 0.62 & 0.688 \\
            & DFIZ & 0.729 & 0.798 & 0.0 & 0.0 & 0.785 & 0.863 & 0.791 & 0.869 & 0.661 & 0.76 & 0.695 & 0.784 & 0.61 & 0.679 \\
            & DLH & 0.802 & 0.862 & 0.0 & 0.0 & 0.803 & 0.881 & 0.823 & 0.899 & 0.666 & 0.764 & 0.697 & 0.785 & 0.632 & 0.699 \\
            & DPH & 0.745 & 0.814 & 0.0 & 0.0 & 0.787 & 0.867 & 0.801 & 0.878 & 0.659 & 0.758 & 0.697 & 0.785 & 0.615 & 0.684 \\
            & DirichletLM & 0.64 & 0.705 & 0.0 & 0.0 & 0.785 & 0.863 & 0.791 & 0.867 & 0.652 & 0.75 & 0.691 & 0.778 & 0.593 & 0.661 \\
            & Hiemstra\_LM & 0.819 & 0.88 & 0.0 & 0.0 & 0.799 & 0.878 & 0.815 & 0.892 & 0.658 & 0.757 & 0.699 & 0.786 & 0.632 & 0.699 \\
            & LGD & 0.765 & 0.833 & 0.0 & 0.0 & 0.787 & 0.866 & 0.792 & 0.87 & 0.659 & 0.758 & 0.697 & 0.785 & 0.617 & 0.685 \\
            & PL2 & 0.766 & 0.829 & 0.0 & 0.0 & 0.796 & 0.876 & 0.819 & 0.896 & 0.656 & 0.755 & 0.698 & 0.786 & 0.623 & 0.69 \\
            & TF\_IDF & 0.769 & 0.832 & 0.0 & 0.0 & 0.793 & 0.872 & 0.808 & 0.885 & 0.657 & 0.756 & 0.697 & 0.785 & 0.62 & 0.688 \\

        \bottomrule
    \end{tabular}}
    \renewcommand{\arraystretch}{1.0}
\end{table}

The next step in the transfer pipeline was to assign passage scores to the passages of the selected documents from the previous step. This was done to identify the most relevant passages of a document with respect to the actual query associated with the document’s relevance judgment. To achieve this, each passage was treated as an independent query and used to retrieve a document ranking from the original source corpus. Since this study does not aim to optimize or analyze specific retriever systems but rather uses them for passage scoring, a diverse selection of models was tested, as listed in Table~\ref{tab:passage-scoring}. Based on the retrieved document rankings, \texttt{precision@10} and \texttt{nDCG@10} were computed and assigned as passage scores.
\\\\
To evaluate the quality of the assigned passage scores and to determine which score type (i.e., which retrieval model and whether \texttt{precision@10} or \texttt{nDCG@10}) is most effective in identifying the best passages, the rank correlation between the original relevance judgments and the newly assigned scores was computed using Kendall's $\tau$ and Spearman's $\rho$. First, an overall document score was aggregated from all its passage scores. For each document, all passage scores were aggregated using one of the following methods: \texttt{mean}, \texttt{min}, \texttt{max}, or \texttt{sum}. Thereafter, the resulting document scores were transformed by applying one of the following transformations: \texttt{id}, \texttt{log}, \texttt{exp}, or \texttt{sqrt}. All 16 combinations of aggregation and transformation were tested across all 10 retrieval models, using both \texttt{precision@10} and \texttt{nDCG@10} scores, and then evaluated with Kendall's $\tau$ and Spearman's $\rho$. These rank correlations were computed for each query individually rather than across all relevance judgments of a retrieval task simultaneously. The intent behind macro-averaging across all queries of a retrieval task was to reduce the effect of outliers for individual queries.
\\\\
To ensure a robust evaluation of the assigned passage scores, a 5-fold cross-validation was applied across the metrics, aggregation methods, and transformation methods for each combination of retrieval model and retrieval task. First, queries within each retrieval task were divided into five equal subsets (folds). For each fold, one subset was designated as the validation set, while the remaining four subsets formed the training set. In each iteration of the cross-validation, the metric, aggregation method, and transformation method that achieved the highest average correlation score on the training set (four-fifths of the queries) were selected. This selection process involved computing the average correlation score across all queries in the training set for each combination. The chosen combination was then evaluated on the test set (the remaining fifth of the queries) to obtain a final correlation score for that iteration. This process was repeated for all five folds, ensuring that each query subset served as the validation set exactly once. The final cross-validation score for each retrieval model and retrieval task was computed as the average correlation score across all five folds, thereby reducing the impact of overfitting and providing a more reliable estimate of passage scoring performance.
\\\\
The results in Table~\ref{tab:passage-scoring} show that a very high rank correlation was achieved for all retrieval models across all tested information retrieval tasks. While \texttt{Hiemstra\_LM} was the best-performing retrieval model, all models demonstrated strong performance when comparing the average $\overline{\tau}$ and $\overline{\rho}$ values. Therefore, the choice of retrieval model is not a critical factor for subsequent steps in the transfer pipeline, as all models produced passage scores with high rank correlation to the actual relevance labels from the retrieval tasks. However, the choice of metric is a significant factor. \texttt{precision@10} resulted in significantly lower rank correlation than \texttt{nDCG@10}. This outcome is expected, as \texttt{precision@10} lacks the granularity needed to differentiate between individual passage scores effectively. Therefore, the clear winning metric for passage scoring is \texttt{nDCG@10}.
\pagebreak


% ===================
% Candidate Selection
% ===================
\section{Candidate Selection}\label{eval-candidate-selection}

\begin{table}[t]
    \centering
    \footnotesize
    % \caption{\texttt{Recall} for the retrieved candidates from the target corpus, along with the total number of candidates. The retrieval process allows multiple passages from the same document in the source dataset.}
    % \caption{Overview of the recall (Rec.) and the total number of candidate documents (Docs.) for all three candidate retrieval approaches. Each approach was tested with restricting the selection of documents to a maximum of one passage per document (opd.) and without (def.). Also, the nearest neighbor, and therefore the union appproach which combines the results of naive and nearest neighbor retrieval, were evaluated with different values of $k$ which specifies the number of used passages for candidate retrievel.}
    \caption{Overview of recall (Rec.) and the total number of candidate documents (Docs.) for all three candidate retrieval approaches, tested with document selection restricted to one passage per document (opd.) and without restriction (def.). The \texttt{Nearest Neighbor} and the \texttt{Union} approaches were also evaluated with varying $k$, determining the number of passages used for candidate retrieval.}
    \resizebox{\textwidth}{!}{%
    \renewcommand{\arraystretch}{1.2}
    \begin{tabular}{ccrcrcrcrcrcrcr}
        \toprule

        \multicolumn{3}{c}{\textbf{Approach}} & \multicolumn{2}{c}{\textbf{Touché 20}} & \multicolumn{2}{c}{\textbf{Robust04}} & \multicolumn{2}{c}{\textbf{TREC-7}} & \multicolumn{2}{c}{\textbf{TREC-8}} & \multicolumn{2}{c}{\textbf{TREC-19 DL}} & \multicolumn{2}{c}{\textbf{TREC-20 DL}} \\

        \cmidrule(lr){4-5} \cmidrule(lr){6-7} \cmidrule(lr){8-9} \cmidrule(lr){10-11} \cmidrule(lr){12-13} \cmidrule(lr){14-15}
                                          & & $k$ & Rec. & Docs. & Rec. & Docs. & Rec. & Docs. & Rec. & Docs. & Rec. & Docs. & Rec. & Docs. \\
        \midrule
        \multirow{2}{*}{\rotatebox{90}{\textbf{Naive}}} & \multirow{2}{*}{-} & \multirow{2}{*}{-} & \multirow{2}{*}{$0.912$} & \multirow{2}{*}{$76\,284$} & \multirow{2}{*}{$0.0$} & \multirow{2}{*}{$0.0$} & \multirow{2}{*}{$0.657$} & \multirow{2}{*}{$80\,032$} & \multirow{2}{*}{$0.713$} & \multirow{2}{*}{$78\,334$} & \multirow{2}{*}{$0.736$} & \multirow{2}{*}{$42\,005$} & \multirow{2}{*}{$0.751$} & \multirow{2}{*}{$51\,575$} \\
        & & & & & & & & & & & & & & \\

        \midrule
        \multirow{6}{*}{\rotatebox{90}{\textbf{Nearest Neighbor}}} & \multirow{3}{*}{def.} &  10 & $0.609$ & $4\,795$ & $0.0$ & $0.0$ & $0.43$ & $4\,286$ & $0.449$ & $4\,488$ & $0.56$ & $3\,521$ & $0.616$ & $4\,493$ \\
                                                                                        &  &  50 & $0.828$ & $15\,603$ & $0.0$ & $0.0$ & $0.696$ & $17\,451$ & $0.665$ & $17\,810$ & $0.872$ & $11\,552$ & $0.906$ & $14\,572$ \\
                                                                                        &  & 100 & $0.978$ & $23\,492$ & $0.0$ & $0.0$ & $0.787$ & $32\,879$ & $0.757$ & $34\,233$ & $0.966$ & $18\,765$ & $0.974$ & $21\,996$ \\
       \cmidrule(lr){2-15}
                                                                   & \multirow{3}{*}{opd.} &  10 & $0.699$ & $5\,105$ & $0.0$ & $0.0$ & $0.495$ & $4\,956$ & $0.489$ & $5\,071$ & $0.56$ & $3\,521$ & $0.616$ & $4\,493$ \\
                                                                                        &  &  50 & $1.0$ & $10\,259$ & $0.0$ & $0.0$ & $0.826$ & $20\,981$ & $0.797$ & $21\,991$ & $0.872$ & $11\,552$ & $0.906$ & $14\,572$ \\
                                                                                        &  & 100 & $1.0$ & $10\,259$ & $0.0$ & $0.0$ & $0.826$ & $20\,981$ & $0.797$ & $21\,991$ & $0.966$ & $18\,765$ & $0.974$ & $21\,996$ \\

        \midrule
        \multirow{6}{*}{\rotatebox{90}{\textbf{Union}}} & \multirow{3}{*}{def.} &  10 & $0.931$ & $77\,306$ & $0.0$ & $0.0$ & $0.771$ & $82\,593$ & $0.795$ & $80\,570$ & $0.859$ & $43\,806$ & $0.856$ & $53\,819$ \\
                                                                             &  &  50 & $0.948$ & $82\,254$ & $0.0$ & $0.0$ & $0.861$ & $93\,008$ & $0.86$ & $90\,699$ & $0.954$ & $49\,298$ & $0.958$ & $61\,137$ \\
                                                                             &  & 100 & $0.992$ & $86\,973$ & $0.0$ & $0.0$ & $0.904$ & $106\,738$ & $0.895$ & $105\,048$ & $0.988$ & $54\,962$ & $0.99$ & $67\,259$ \\
        \cmidrule(lr){2-15}
                                                        & \multirow{3}{*}{opd.} &  10 & $0.935$ & $77\,204$ & $0.0$ & $0.0$ & $0.792$ & $82\,983$ & $0.808$ & $81\,071$ & $0.859$ & $43\,806$ & $0.856$ & $53\,819$ \\
                                                                             &  &  50 &  $1.0$ & $78\,918$ & $0.0$ & $0.0$ & $0.915$ & $95\,849$ & $0.919$ & $94\,402$ & $0.954$ & $49\,298$ & $0.958$ & $61\,137$ \\
                                                                             &  & 100 & $1.0$ & $78\,918$ & $0.0$ & $0.0$ & $0.915$ & $95\,849$ & $0.919$ & $94\,402$ & $0.988$ & $54\,962$ & $0.99$ & $67\,259$ \\

        \bottomrule 
    \end{tabular}}
    \renewcommand{\arraystretch}{1.0}
\end{table}


% ====================
% Pairwise Preferences
% ====================
\section{Pairwise Preferences}\label{eval-pairwise-preferences}

The first evaluation of the pairwise preferences involves inferring relevance scores from the original relevance judgments within the source dataset itself. This allows for an automatic evaluation using rank correlation, as the retrieval tasks provide predefined relevance judgments for comparison. Each dataset can be evaluated individually to verify whether the inferred scores align with the original relevance judgments. The second evaluation assesses the transfer of inferred relevance scores from the retrieval tasks to the target corpus, \texttt{ClueWeb22/b}. Since no existing relevance judgments are available for \texttt{ClueWeb22/b}, this evaluation cannot be automated. Instead, randomly selected documents for which relevance scores have been inferred are chosen for each query and manually reviewed. Human assessors then determine whether the inferred relevance scores accurately reflect the relevance of the corresponding documents to the queries.

% \begin{table}[t]
%     \centering
%     \footnotesize
%     \caption{Correlations of the inferred relevance judgements to the original document judgments reported in terms of Kendall's $\tau$ and Spearman's $\rho$.}
%     \resizebox{\textwidth}{!}{%
%     \renewcommand{\arraystretch}{1.0}
%     \begin{tabular}{cclccccccccccccc}
%         \toprule

%         \multicolumn{3}{c}{\textbf{Approach}} & \multicolumn{2}{c}{\textbf{Touché 20}} & \multicolumn{2}{c}{\textbf{Robust04}} & \multicolumn{2}{c}{\textbf{TREC-7}} & \multicolumn{2}{c}{\textbf{TREC-8}} & \multicolumn{2}{c}{\textbf{TREC-19 DL}} & \multicolumn{2}{c}{\textbf{TREC-20 DL}} \\
%         \cmidrule(lr){4-5} \cmidrule(lr){6-7} \cmidrule(lr){8-9} \cmidrule(lr){10-11} \cmidrule(lr){12-13} \cmidrule(lr){14-15}
%                             & & & $\tau$ & $\rho$ & $\tau$ & $\rho$ & $\tau$ & $\rho$ & $\tau$ & $\rho$ & $\tau$ & $\rho$ & $\tau$ & $\rho$ \\
%         \midrule
%         \multirow{16}{*}{\rotatebox{90}{\makecell{\textbf{Pairwise Preferences} \\ \textbf{flant5-base}}}}
%             & \multirow{4}{*}{\textbf{mean}} 
%               & \textbf{id}   & 0.000 & 0.000 & 0.000 & 0.000 & 0.000 & 0.000 & 0.000 & 0.000 & 0.000 & 0.000 & 0.000 & 0.000 \\
%             & & \textbf{log}  & 0.000 & 0.000 & 0.000 & 0.000 & 0.000 & 0.000 & 0.000 & 0.000 & 0.000 & 0.000 & 0.000 & 0.000 \\
%             & & \textbf{exp}  & 0.000 & 0.000 & 0.000 & 0.000 & 0.000 & 0.000 & 0.000 & 0.000 & 0.000 & 0.000 & 0.000 & 0.000 \\
%             & & \textbf{sqrt} & 0.000 & 0.000 & 0.000 & 0.000 & 0.000 & 0.000 & 0.000 & 0.000 & 0.000 & 0.000 & 0.000 & 0.000 \\
%         \cmidrule{2-15}
%             & \multirow{4}{*}{\textbf{min}} 
%               & \textbf{id}   & 0.000 & 0.000 & 0.000 & 0.000 & 0.000 & 0.000 & 0.000 & 0.000 & 0.000 & 0.000 & 0.000 & 0.000 \\
%             & & \textbf{log}  & 0.000 & 0.000 & 0.000 & 0.000 & 0.000 & 0.000 & 0.000 & 0.000 & 0.000 & 0.000 & 0.000 & 0.000 \\
%             & & \textbf{exp}  & 0.000 & 0.000 & 0.000 & 0.000 & 0.000 & 0.000 & 0.000 & 0.000 & 0.000 & 0.000 & 0.000 & 0.000 \\
%             & & \textbf{sqrt} & 0.000 & 0.000 & 0.000 & 0.000 & 0.000 & 0.000 & 0.000 & 0.000 & 0.000 & 0.000 & 0.000 & 0.000 \\
%         \cmidrule{2-15}
%             & \multirow{4}{*}{\textbf{max}} 
%               & \textbf{id}   & 0.000 & 0.000 & 0.000 & 0.000 & 0.000 & 0.000 & 0.000 & 0.000 & 0.000 & 0.000 & 0.000 & 0.000 \\
%             & & \textbf{log}  & 0.000 & 0.000 & 0.000 & 0.000 & 0.000 & 0.000 & 0.000 & 0.000 & 0.000 & 0.000 & 0.000 & 0.000 \\
%             & & \textbf{exp}  & 0.000 & 0.000 & 0.000 & 0.000 & 0.000 & 0.000 & 0.000 & 0.000 & 0.000 & 0.000 & 0.000 & 0.000 \\
%             & & \textbf{sqrt} & 0.000 & 0.000 & 0.000 & 0.000 & 0.000 & 0.000 & 0.000 & 0.000 & 0.000 & 0.000 & 0.000 & 0.000 \\
%         \cmidrule{2-15}
%             & \multirow{4}{*}{\textbf{sum}} 
%               & \textbf{id}   & 0.000 & 0.000 & 0.000 & 0.000 & 0.000 & 0.000 & 0.000 & 0.000 & 0.000 & 0.000 & 0.000 & 0.000 \\
%             & & \textbf{log}  & 0.000 & 0.000 & 0.000 & 0.000 & 0.000 & 0.000 & 0.000 & 0.000 & 0.000 & 0.000 & 0.000 & 0.000 \\
%             & & \textbf{exp}  & 0.000 & 0.000 & 0.000 & 0.000 & 0.000 & 0.000 & 0.000 & 0.000 & 0.000 & 0.000 & 0.000 & 0.000 \\
%             & & \textbf{sqrt} & 0.000 & 0.000 & 0.000 & 0.000 & 0.000 & 0.000 & 0.000 & 0.000 & 0.000 & 0.000 & 0.000 & 0.000 \\
%         \midrule

%         \multirow{16}{*}{\rotatebox{90}{\makecell{\textbf{Pairwise Preferences} \\ \textbf{flant5-small}}}}
%             & \multirow{4}{*}{\textbf{mean}} 
%               & \textbf{id}   & 0.000 & 0.000 & 0.000 & 0.000 & 0.000 & 0.000 & 0.000 & 0.000 & 0.000 & 0.000 & 0.000 & 0.000 \\
%             & & \textbf{log}  & 0.000 & 0.000 & 0.000 & 0.000 & 0.000 & 0.000 & 0.000 & 0.000 & 0.000 & 0.000 & 0.000 & 0.000 \\
%             & & \textbf{exp}  & 0.000 & 0.000 & 0.000 & 0.000 & 0.000 & 0.000 & 0.000 & 0.000 & 0.000 & 0.000 & 0.000 & 0.000 \\
%             & & \textbf{sqrt} & 0.000 & 0.000 & 0.000 & 0.000 & 0.000 & 0.000 & 0.000 & 0.000 & 0.000 & 0.000 & 0.000 & 0.000 \\
%         \cmidrule{2-15}
%             & \multirow{4}{*}{\textbf{min}} 
%               & \textbf{id}   & 0.000 & 0.000 & 0.000 & 0.000 & 0.000 & 0.000 & 0.000 & 0.000 & 0.000 & 0.000 & 0.000 & 0.000 \\
%             & & \textbf{log}  & 0.000 & 0.000 & 0.000 & 0.000 & 0.000 & 0.000 & 0.000 & 0.000 & 0.000 & 0.000 & 0.000 & 0.000 \\
%             & & \textbf{exp}  & 0.000 & 0.000 & 0.000 & 0.000 & 0.000 & 0.000 & 0.000 & 0.000 & 0.000 & 0.000 & 0.000 & 0.000 \\
%             & & \textbf{sqrt} & 0.000 & 0.000 & 0.000 & 0.000 & 0.000 & 0.000 & 0.000 & 0.000 & 0.000 & 0.000 & 0.000 & 0.000 \\
%         \cmidrule{2-15}
%             & \multirow{4}{*}{\textbf{max}} 
%               & \textbf{id}   & 0.000 & 0.000 & 0.000 & 0.000 & 0.000 & 0.000 & 0.000 & 0.000 & 0.000 & 0.000 & 0.000 & 0.000 \\
%             & & \textbf{log}  & 0.000 & 0.000 & 0.000 & 0.000 & 0.000 & 0.000 & 0.000 & 0.000 & 0.000 & 0.000 & 0.000 & 0.000 \\
%             & & \textbf{exp}  & 0.000 & 0.000 & 0.000 & 0.000 & 0.000 & 0.000 & 0.000 & 0.000 & 0.000 & 0.000 & 0.000 & 0.000 \\
%             & & \textbf{sqrt} & 0.000 & 0.000 & 0.000 & 0.000 & 0.000 & 0.000 & 0.000 & 0.000 & 0.000 & 0.000 & 0.000 & 0.000 \\
%         \cmidrule{2-15}
%             & \multirow{4}{*}{\textbf{sum}} 
%               & \textbf{id}   & 0.000 & 0.000 & 0.000 & 0.000 & 0.000 & 0.000 & 0.000 & 0.000 & 0.000 & 0.000 & 0.000 & 0.000 \\
%             & & \textbf{log}  & 0.000 & 0.000 & 0.000 & 0.000 & 0.000 & 0.000 & 0.000 & 0.000 & 0.000 & 0.000 & 0.000 & 0.000 \\
%             & & \textbf{exp}  & 0.000 & 0.000 & 0.000 & 0.000 & 0.000 & 0.000 & 0.000 & 0.000 & 0.000 & 0.000 & 0.000 & 0.000 \\
%             & & \textbf{sqrt} & 0.000 & 0.000 & 0.000 & 0.000 & 0.000 & 0.000 & 0.000 & 0.000 & 0.000 & 0.000 & 0.000 & 0.000 \\
%         \midrule

%         \multirow{16}{*}{\rotatebox{90}{\makecell{\textbf{Pairwise Preferences} \\ \textbf{t5-small}}}}
%             & \multirow{4}{*}{\textbf{mean}} 
%               & \textbf{id}   & 0.000 & 0.000 & 0.000 & 0.000 & 0.000 & 0.000 & 0.000 & 0.000 & 0.000 & 0.000 & 0.000 & 0.000 \\
%             & & \textbf{log}  & 0.000 & 0.000 & 0.000 & 0.000 & 0.000 & 0.000 & 0.000 & 0.000 & 0.000 & 0.000 & 0.000 & 0.000 \\
%             & & \textbf{exp}  & 0.000 & 0.000 & 0.000 & 0.000 & 0.000 & 0.000 & 0.000 & 0.000 & 0.000 & 0.000 & 0.000 & 0.000 \\
%             & & \textbf{sqrt} & 0.000 & 0.000 & 0.000 & 0.000 & 0.000 & 0.000 & 0.000 & 0.000 & 0.000 & 0.000 & 0.000 & 0.000 \\
%         \cmidrule{2-15}
%             & \multirow{4}{*}{\textbf{min}} 
%               & \textbf{id}   & 0.000 & 0.000 & 0.000 & 0.000 & 0.000 & 0.000 & 0.000 & 0.000 & 0.000 & 0.000 & 0.000 & 0.000 \\
%             & & \textbf{log}  & 0.000 & 0.000 & 0.000 & 0.000 & 0.000 & 0.000 & 0.000 & 0.000 & 0.000 & 0.000 & 0.000 & 0.000 \\
%             & & \textbf{exp}  & 0.000 & 0.000 & 0.000 & 0.000 & 0.000 & 0.000 & 0.000 & 0.000 & 0.000 & 0.000 & 0.000 & 0.000 \\
%             & & \textbf{sqrt} & 0.000 & 0.000 & 0.000 & 0.000 & 0.000 & 0.000 & 0.000 & 0.000 & 0.000 & 0.000 & 0.000 & 0.000 \\
%         \cmidrule{2-15}
%             & \multirow{4}{*}{\textbf{max}} 
%               & \textbf{id}   & 0.000 & 0.000 & 0.000 & 0.000 & 0.000 & 0.000 & 0.000 & 0.000 & 0.000 & 0.000 & 0.000 & 0.000 \\
%             & & \textbf{log}  & 0.000 & 0.000 & 0.000 & 0.000 & 0.000 & 0.000 & 0.000 & 0.000 & 0.000 & 0.000 & 0.000 & 0.000 \\
%             & & \textbf{exp}  & 0.000 & 0.000 & 0.000 & 0.000 & 0.000 & 0.000 & 0.000 & 0.000 & 0.000 & 0.000 & 0.000 & 0.000 \\
%             & & \textbf{sqrt} & 0.000 & 0.000 & 0.000 & 0.000 & 0.000 & 0.000 & 0.000 & 0.000 & 0.000 & 0.000 & 0.000 & 0.000 \\
%         \cmidrule{2-15}
%             & \multirow{4}{*}{\textbf{sum}} 
%               & \textbf{id}   & 0.000 & 0.000 & 0.000 & 0.000 & 0.000 & 0.000 & 0.000 & 0.000 & 0.000 & 0.000 & 0.000 & 0.000 \\
%             & & \textbf{log}  & 0.000 & 0.000 & 0.000 & 0.000 & 0.000 & 0.000 & 0.000 & 0.000 & 0.000 & 0.000 & 0.000 & 0.000 \\
%             & & \textbf{exp}  & 0.000 & 0.000 & 0.000 & 0.000 & 0.000 & 0.000 & 0.000 & 0.000 & 0.000 & 0.000 & 0.000 & 0.000 \\
%             & & \textbf{sqrt} & 0.000 & 0.000 & 0.000 & 0.000 & 0.000 & 0.000 & 0.000 & 0.000 & 0.000 & 0.000 & 0.000 & 0.000 \\
%         \midrule

%         \multirow{5}{*}{\rotatebox{90}{\makecell{\textbf{Pointwise} \\ \textbf{Preferences}}}}
%             & & & & & & & & & & & & & & \\
%             & \multicolumn{2}{c}{\textbf{flant5-base}}   & 0.000 & 0.000 & 0.000 & 0.000 & 0.000 & 0.000 & 0.000 & 0.000 & 0.000 & 0.000 & 0.000 & 0.000 \\
%             & \multicolumn{2}{c}{\textbf{flant5-small}}   & 0.000 & 0.000 & 0.000 & 0.000 & 0.000 & 0.000 & 0.000 & 0.000 & 0.000 & 0.000 & 0.000 & 0.000 \\
%             & \multicolumn{2}{c}{\textbf{t5-small}}   & 0.000 & 0.000 & 0.000 & 0.000 & 0.000 & 0.000 & 0.000 & 0.000 & 0.000 & 0.000 & 0.000 & 0.000 \\
%             & & & & & & & & & & & & & & \\
%         \bottomrule 
%     \end{tabular}}
%     \renewcommand{\arraystretch}{1.0}
% \end{table}

\begin{table}[t]
  \centering
  \footnotesize
  \caption{Correlations of the inferred relevance judgements to the original document judgments reported in terms of Kendall's $\tau$ and Spearman's $\rho$.}
  \resizebox{\textwidth}{!}{%
  \renewcommand{\arraystretch}{1.0}
  \begin{tabular}{cclccccccccccc}
      \toprule

      \multicolumn{3}{c}{\textbf{Approach}} & \multicolumn{2}{c}{\textbf{Touché 20}} & \multicolumn{2}{c}{\textbf{TREC-7}} & \multicolumn{2}{c}{\textbf{TREC-8}} & \multicolumn{2}{c}{\textbf{TREC-19 DL}} & \multicolumn{2}{c}{\textbf{TREC-20 DL}} \\
      \cmidrule(lr){4-5} \cmidrule(lr){6-7} \cmidrule(lr){8-9} \cmidrule(lr){10-11} \cmidrule(lr){12-13}
                          & & & $\tau$ & $\rho$ & $\tau$ & $\rho$ & $\tau$ & $\rho$ & $\tau$ & $\rho$ & $\tau$ & $\rho$ \\
      \midrule
      \multirow{16}{*}{\rotatebox{90}{\makecell{\textbf{Pairwise Preferences} \\ \textbf{flant5-base}}}}
      & \multirow{4}{*}{\textbf{mean}}
      & \textbf{id} & 0.218 & 0.271 & 0.117 & 0.143 & 0.168 & 0.205 & 0.159 & 0.199 & 0.218 & 0.270\\
    & & \textbf{log} & 0.218 & 0.271 & 0.117 & 0.143 & 0.168 & 0.205 & 0.159 & 0.199 & 0.218 & 0.270\\
    & & \textbf{exp} & 0.218 & 0.271 & 0.117 & 0.143 & 0.168 & 0.205 & 0.159 & 0.199 & 0.218 & 0.270\\
    & & \textbf{sqrt} & 0.218 & 0.271 & 0.117 & 0.143 & 0.168 & 0.205 & 0.159 & 0.199 & 0.218 & 0.270\\
\cmidrule{2-13}
& \multirow{4}{*}{\textbf{min}}
      & \textbf{id} & 0.233 & 0.293 & 0.125 & 0.153 & 0.184 & 0.225 & 0.184 & 0.231 & 0.225 & 0.279 \\
    & & \textbf{log} & 0.233 & 0.293 & 0.125 & 0.153 & 0.184 & 0.225 & 0.184 & 0.231 & 0.225 & 0.279 \\
    & & \textbf{exp} & 0.233 & 0.293 & 0.125 & 0.153 & 0.184 & 0.225 & 0.184 & 0.231 & 0.225 & 0.279 \\
    & & \textbf{sqrt} & 0.233 & 0.293 & 0.125 & 0.153 & 0.184 & 0.225 & 0.184 & 0.231 & 0.225 & 0.279 \\
\cmidrule{2-13}
& \multirow{4}{*}{\textbf{max}}
      & \textbf{id} & 0.222 & 0.275 & 0.087 & 0.106 & 0.074 & 0.090& 0.077 & 0.097 & 0.100& 0.123 \\
    & & \textbf{log} & 0.222 & 0.275 & 0.087 & 0.106 & 0.074 & 0.090& 0.077 & 0.097 & 0.100& 0.123 \\
    & & \textbf{exp} & 0.222 & 0.275 & 0.087 & 0.106 & 0.074 & 0.090& 0.077 & 0.097 & 0.100& 0.123 \\
    & & \textbf{sqrt} & 0.222 & 0.275 & 0.087 & 0.106 & 0.074 & 0.090& 0.077 & 0.097 & 0.100& 0.123 \\
\cmidrule{2-13}
& \multirow{4}{*}{\textbf{sum}}
      & \textbf{id} & 0.218 & 0.271 & 0.117 & 0.143 & 0.168 & 0.205 & 0.159 & 0.199 & 0.218 & 0.270\\
    & & \textbf{log} & 0.218 & 0.271 & 0.117 & 0.143 & 0.168 & 0.205 & 0.159 & 0.199 & 0.218 & 0.270\\
    & & \textbf{exp} & 0.218 & 0.271 & 0.117 & 0.143 & 0.168 & 0.205 & 0.159 & 0.199 & 0.218 & 0.270\\
    & & \textbf{sqrt} & 0.218 & 0.271 & 0.117 & 0.143 & 0.168 & 0.205 & 0.159 & 0.199 & 0.218 & 0.270\\
      \midrule

      \multirow{16}{*}{\rotatebox{90}{\makecell{\textbf{Pairwise Preferences} \\ \textbf{flant5-small}}}}
      & \multirow{4}{*}{\textbf{mean}}
      & \textbf{id} & 0.253 & 0.324 & 0.056 & 0.069 & 0.073 & 0.089 & 0.142 & 0.179 & 0.136 & 0.169 \\
    & & \textbf{log} & 0.253 & 0.324 & 0.056 & 0.069 & 0.073 & 0.089 & 0.142 & 0.179 & 0.136 & 0.169 \\
    & & \textbf{exp} & 0.253 & 0.324 & 0.056 & 0.069 & 0.073 & 0.089 & 0.142 & 0.179 & 0.136 & 0.169 \\
    & & \textbf{sqrt} & 0.253 & 0.324 & 0.056 & 0.069 & 0.073 & 0.089 & 0.142 & 0.179 & 0.136 & 0.169 \\
\cmidrule{2-13}
& \multirow{4}{*}{\textbf{min}}
      & \textbf{id} & 0.300& 0.383 & 0.072 & 0.088 & 0.090& 0.110& 0.142 & 0.178 & 0.154 & 0.191 \\
    & & \textbf{log} & 0.300& 0.383 & 0.072 & 0.088 & 0.090& 0.110& 0.142 & 0.178 & 0.154 & 0.191 \\
    & & \textbf{exp} & 0.300& 0.383 & 0.072 & 0.088 & 0.090& 0.110& 0.142 & 0.178 & 0.154 & 0.191 \\
    & & \textbf{sqrt} & 0.300& 0.383 & 0.072 & 0.088 & 0.090& 0.110& 0.142 & 0.178 & 0.154 & 0.191 \\
\cmidrule{2-13}
& \multirow{4}{*}{\textbf{max}}
      & \textbf{id} & 0.200& 0.254 & 0.028 & 0.034 & 0.033 & 0.041 & 0.050& 0.062 & 0.081 & 0.100\\
    & & \textbf{log} & 0.200& 0.254 & 0.028 & 0.034 & 0.033 & 0.041 & 0.050& 0.062 & 0.081 & 0.100\\
    & & \textbf{exp} & 0.200& 0.254 & 0.028 & 0.034 & 0.033 & 0.041 & 0.050& 0.062 & 0.081 & 0.100\\
    & & \textbf{sqrt} & 0.200& 0.254 & 0.028 & 0.034 & 0.033 & 0.041 & 0.050& 0.062 & 0.081 & 0.100\\
\cmidrule{2-13}
& \multirow{4}{*}{\textbf{sum}}
      & \textbf{id} & 0.253 & 0.324 & 0.056 & 0.069 & 0.073 & 0.089 & 0.142 & 0.179 & 0.136 & 0.169 \\
    & & \textbf{log} & 0.253 & 0.324 & 0.056 & 0.069 & 0.073 & 0.089 & 0.142 & 0.179 & 0.136 & 0.169 \\
    & & \textbf{exp} & 0.253 & 0.324 & 0.056 & 0.069 & 0.073 & 0.089 & 0.142 & 0.179 & 0.136 & 0.169 \\
    & & \textbf{sqrt} & 0.253 & 0.324 & 0.056 & 0.069 & 0.073 & 0.089 & 0.142 & 0.179 & 0.136 & 0.169 \\
      \midrule

      \multirow{16}{*}{\rotatebox{90}{\makecell{\textbf{Pairwise Preferences} \\ \textbf{t5-small}}}}
      & \multirow{4}{*}{\textbf{mean}}
      & \textbf{id} & 0.250& 0.322 & 0.000& 0.000& -0.008 & -0.010& 0.003 & 0.005 & -0.001 & -0.001 \\
    & & \textbf{log} & 0.250& 0.322 & 0.000& 0.000& -0.008 & -0.010& 0.003 & 0.005 & -0.001 & -0.001 \\
    & & \textbf{exp} & 0.250& 0.322 & 0.000& 0.000& -0.008 & -0.010& 0.003 & 0.005 & -0.001 & -0.001 \\
    & & \textbf{sqrt} & 0.250& 0.322 & 0.000& 0.000& -0.008 & -0.010& 0.003 & 0.005 & -0.001 & -0.001 \\
\cmidrule{2-13}
& \multirow{4}{*}{\textbf{min}}
      & \textbf{id} & 0.310& 0.400& 0.017 & 0.021 & 0.005 & 0.007 & 0.007 & 0.010& 0.013 & 0.016 \\
    & & \textbf{log} & 0.310& 0.400& 0.017 & 0.021 & 0.005 & 0.007 & 0.007 & 0.010& 0.013 & 0.016 \\
    & & \textbf{exp} & 0.310& 0.400& 0.017 & 0.021 & 0.005 & 0.007 & 0.007 & 0.010& 0.013 & 0.016 \\
    & & \textbf{sqrt} & 0.310& 0.400& 0.017 & 0.021 & 0.005 & 0.007 & 0.007 & 0.010& 0.013 & 0.016 \\
\cmidrule{2-13}
& \multirow{4}{*}{\textbf{max}}
      & \textbf{id} & 0.164 & 0.213 & 0.004 & 0.005 & -0.005 & -0.007 & 0.005 & 0.007 & 0.008 & 0.010\\
    & & \textbf{log} & 0.164 & 0.213 & 0.004 & 0.005 & -0.005 & -0.007 & 0.005 & 0.007 & 0.008 & 0.010\\
    & & \textbf{exp} & 0.164 & 0.213 & 0.004 & 0.005 & -0.005 & -0.007 & 0.005 & 0.007 & 0.008 & 0.010\\
    & & \textbf{sqrt} & 0.164 & 0.213 & 0.004 & 0.005 & -0.005 & -0.007 & 0.005 & 0.007 & 0.008 & 0.010\\
\cmidrule{2-13}
& \multirow{4}{*}{\textbf{sum}}
      & \textbf{id} & 0.250& 0.322 & 0.000& 0.000& -0.008 & -0.010& 0.003 & 0.005 & -0.001 & -0.001 \\
    & & \textbf{log} & 0.250& 0.322 & 0.000& 0.000& -0.008 & -0.010& 0.003 & 0.005 & -0.001 & -0.001 \\
    & & \textbf{exp} & 0.250& 0.322 & 0.000& 0.000& -0.008 & -0.010& 0.003 & 0.005 & -0.001 & -0.001 \\
    & & \textbf{sqrt} & 0.250& 0.322 & 0.000& 0.000& -0.008 & -0.010& 0.003 & 0.005 & -0.001 & -0.001 \\
      \midrule

      \multirow{5}{*}{\rotatebox{90}{\makecell{\textbf{Pointwise} \\ \textbf{Preferences}}}}
          & & & & & & & & & & & & \\
          & \multicolumn{2}{c}{\textbf{flant5-base}}   & 0.222 & 0.275 & 0.087 & 0.106 & 0.074 & 0.090& 0.077 & 0.097 & 0.100& 0.123 \\
          & \multicolumn{2}{c}{\textbf{flant5-small}}   & 0.200& 0.254 & 0.028 & 0.034 & 0.033 & 0.041 & 0.050& 0.062 & 0.081 & 0.100 \\
          & \multicolumn{2}{c}{\textbf{t5-small}}   & 0.164 & 0.213 & 0.004 & 0.005 & -0.005 & -0.007 & 0.005 & 0.007 & 0.008 & 0.010\\
          & & & & & & & & & & & & \\
      \bottomrule 
  \end{tabular}}
  \renewcommand{\arraystretch}{1.0}
\end{table}

% Transfer Pipeline to Source Corpus itself
\subsection{Transfer Pipeline on Source Corpus}\label{eval-pairwise-preferences-source}

% Transfer Pipeline to ClueWeb22/b
\subsection{Transfer Pipeline on \texttt{ClueWeb22/b}}\label{eval-pairwise-preferences-target}