\chapter{Evaluation}\label{evaluation}

This chapter focuses on the evaluation of the transfer pipeline. First, the individual steps of the pipeline are analyzed and evaluated to examine intermediate results and identify potential weaknesses, starting with an assessment of the scores assigned to the selected passages from the source datasets. Next, the different approaches for selecting candidate documents from the target document corpus are evaluated to determine the most effective approach.
\\\\
The evaluation of the candidate retrieval approaches can only be conducted by applying the transfer pipeline to a target dataset that already contains query relevance judgments. This is necessary because the goal of the candidate retrieval is to pre-select candidate documents from a target corpus that are likely to be relevant to a query. To measure this accurately, the target documents must be labeled with existing relevance judgments.
\\\\
The second part of the chapter focuses on the evaluation of the inferred relevance judgments through two transfer strategies. The first evaluation applies the transfer pipeline to a source dataset and uses the same dataset as the target. This self-transfer evaluation measures how well the pipeline reproduces known relevance judgments within the same dataset. The second evaluation assesses the final transfer to \texttt{ClueWeb22/b} as the target dataset. To facilitate this evaluation, a pooling process is conducted on the candidate documents selected by the best-performing candidate retrieval approach. This is followed by a manual relevance judgment of the pooled documents. The quality of the transferred relevance judgments to \texttt{ClueWeb22/b} is then evaluated based on these manual relevance judgments.
% ================
% Rank Correlation
% ================
\section{Rank Correlation}\label{rank-correlation}

Before starting the actual evaluation of the transfer pipeline, I present the metrics used to evaluate the relevance judgments produced by the pipeline against the actual relevance judgments provided by the retrieval tasks. This clarification is important, as the evaluation results of the passage scores (Section~\ref{passage-scoring}) and the final relevance judgments (Section~\ref{pairwise-preferences}) generated by the transfer pipeline are based on these metrics.
\\\\
The \texttt{Inter-Annotator Agreement} (\texttt{IAA})~\citep{artstein:2017} is a statistical measure that quantifies the consistency between annotations provided by multiple annotators. It indicates the level of agreement or disagreement among annotators when labeling the same dataset, thereby offering insight into the quality of the annotations. In this thesis, the ground truth annotation is represented by the relevance judgments provided by the retrieval tasks. These judgments (annotations) are compared to the inferred relevance judgments generated by the transfer pipeline. A high \texttt{IAA} score indicates a strong agreement between the actual relevance labels and the inferred relevance labels.
\\\\
Various methods exist for calculating \texttt{IAA}, each with its strengths and weaknesses. In this thesis, the retrieval tasks provide judgments as integer labels, i.e., $\in \mathbb{N}$, while the inferred relevance labels are represented as actual relevance scores, i.e., $\in \mathbb{R}^{+}$. Consequently, not every correlation method is suitable for computing the aggreement between the relevances. Therefore, the correlation metrics used to evaluate the level of agreement are \texttt{Kendall's $\tau$} and \texttt{Spearman's $\rho$}. Both metrics compute the rank correlation between linear orders~\citep{monjardet:1998}, making them well-suited to the nature of the retrieval tasks labels and inferred relevance scores in this thesis.
\\\\
Table~\ref{tab:rank-correlation} presents an example of how the correlation metrics are calculated based on some sample labels of the relevance judgments. The table shows the agreement between example reference scores $[0.2, 0.7, 0.5]$ and three label sets: the first two with an expected correlation of $1$, and the last one with lower correlation. The \texttt{Default} column represents the actual rank correlation scores computed by the standard algorithms of \texttt{Kendall's $\tau$} and \texttt{Spearman's $\rho$}.
\\\\\\\\\\\\\\
In contrast to the default algorithms, the desired rank correlation for this thesis, is characterized by the ability to map the inferred relevance scores to the actual relevance labels from the retrieval task in the best possible way. For example, the relevance scores $[0.2, 0.7, 0.5]$ should be perfectly mapped to the label set $[0, 1, 1]$ in the second row of Table~\ref{tab:rank-correlation}. This could be achieved by assuming that all relevance scores smaller equal $0.2$ correspond to label $0$ and all scores greater $0.2$ correspond to label $1$. Therefore, the intended rank correlation for this example should be $1$, in this thesis. To achieve this, a modified variant called \texttt{Greedy} was tested alongside the default versions. This variant greedily maps the reference scores to the label set, representing the idealized rank correlation outcome. This example demonstrates that the default correlation metrics are sensitive to the rank order of the relevance judgments, whereas the \texttt{Greedy} variant avoids this sensitivity, achieving the maximum possible correlation scores.


\begin{table}[t]
  \centering
  \caption{Comparison of rank correlation metrics computed by Kendall's $\tau$ and Spearman's $\rho$. The table shows the agreement between the reference scores $[0.2,0.7,0.5]$ and three label sets: two with an expected correlation of $1$, and one with lower correlation. Default represents the actual rank correlation scores, while Greedy shows the scores obtained by greedily mapping the reference scores to the label set, representing the idealized rank correlation outcome.}
  \label{tab:rank-correlation}
  \begin{tabular}{ccccc}
      \toprule
      \textbf{Comparative Set} & \multicolumn{2}{c}{\textbf{Default}} & \multicolumn{2}{c}{\textbf{Greedy}} \\
      \cmidrule(lr){2-3} \cmidrule(lr){4-5}
                               & $\tau$ & $\rho$ & $\tau$ & $\rho$ \\
      \midrule
      
      $[0, 2, 1]$ & 1.00 & 0.99  & 1.00  & 1.00 \\
      $[0, 1, 1]$ & 0.82 & 0.92  & 1.00  & 1.00 \\
      $[0, 0, 1]$ & 0.00 & 0.11  & 0.50  & 0.50 \\
      \bottomrule
  \end{tabular}
\end{table}


% ==================
% Document Selection
% ==================
\section{Document Selection}\label{eval-doucment-selection}

% ===============
% Passage Scoring
% ===============
\section{Passage Scoring}\label{eval-passage-scoring}

Since this study does not aim to optimize or analyze a specific set of retriever systems but uses them for passage scoring, a diverse selection of models was employed to minimize bias in the scores that could arise from relying on a single model. For that reason \texttt{Precision} and \texttt{nDCG} scores were calculated using the following retriever systems: \texttt{BM25, DFR\_BM25, DFIZ, DLH, DPH, DirichletLM, Hiemstra\_LM, LGD, PL2, and TF\_IDF}. An evaluation of which model best scores the passages in relation to the goal of relevance transfer follows in Chapter~\ref{evaluation}~[\nameref{rank-correlation-passage-scores}].

\begin{table}[t]
    \centering
    \caption{Correlations of the assigned passage scores to the original document judgments reported in terms of Kendall's $\tau$ and Spearman's $\rho$.}
    \resizebox{\textwidth}{!}{%
    \renewcommand{\arraystretch}{1.2}
    \begin{tabular}{lcccccccccccccc}
        \toprule
        \textbf{Retrieval Model} & \multicolumn{2}{c}{\textbf{Touché 20}} & \multicolumn{2}{c}{\textbf{Robust04}} & \multicolumn{2}{c}{\textbf{TREC-7}} & \multicolumn{2}{c}{\textbf{TREC-8}} & \multicolumn{2}{c}{\textbf{TREC-19 DL}} & \multicolumn{2}{c}{\textbf{TREC-20 DL}} & & \\
        \cmidrule(lr){2-3} \cmidrule(lr){4-5} \cmidrule(lr){6-7} \cmidrule(lr){8-9} \cmidrule(lr){10-11} \cmidrule(lr){12-13}
                                 & $\tau$ & $\rho$ & $\tau$ & $\rho$ & $\tau$ & $\rho$ & $\tau$ & $\rho$ & $\tau$ & $\rho$ & $\tau$ & $\rho$ & $\overline{\tau}$ & $\overline{\rho}$ \\
        \midrule

        BM25         & 0.809          & 0.882          & 0.835          & 0.907          & 0.834          & 0.904          & 0.823          & 0.906          & 0.741          & 0.867          & 0.751          & 0.860          & 0.799          & 0.888          \\
        DFR\_BM25    & 0.808          & 0.877          & 0.835          & 0.907          & 0.835          & 0.905          & 0.822          & 0.905          & 0.739          & 0.867          & 0.784          & 0.872          & 0.804          & 0.889          \\
        DFIZ         & 0.769          & 0.848          & 0.832          & 0.905          & 0.826          & 0.895          & 0.825          & 0.897          & 0.760          & 0.863          & 0.757          & 0.872          & 0.795          & 0.880          \\
        DLH          & \textbf{0.832} & \textbf{0.909} & 0.832          & \textbf{0.912} & 0.828          & 0.905          & 0.816          & \textbf{0.908} & 0.729          & 0.843          & 0.801          & 0.884          & 0.807          & 0.894          \\
        DPH          & 0.787          & 0.863          & 0.837          & 0.903          & \textbf{0.839} & \textbf{0.910} & 0.828          & 0.904          & 0.714          & 0.852          & 0.752          & 0.847          & 0.793          & 0.880          \\
        DirichletLM  & 0.707          & 0.786          & 0.836          & 0.899          & 0.829          & 0.901          & 0.828          & 0.896          & 0.733          & 0.869          & \textbf{0.818} & \textbf{0.898} & 0.792          & 0.875          \\
        Hiemstra\_LM & 0.830          & 0.901          & 0.838          & \textbf{0.912} & 0.833          & 0.900          & 0.823          & \textbf{0.908} & \textbf{0.773} & \textbf{0.874} & 0.790          & 0.877          & \textbf{0.814} & \textbf{0.895} \\
        LGD          & 0.806          & 0.884          & \textbf{0.840} & 0.908          & 0.838          & \textbf{0.910} & \textbf{0.830} & 0.900          & 0.758          & 0.864          & 0.788          & 0.879          & 0.810          & 0.891          \\
        PL2          & 0.811          & 0.881          & 0.828          & 0.908          & 0.824          & 0.901          & 0.815          & 0.906          & 0.730          & 0.841          & 0.778          & 0.868          & 0.798          & 0.884          \\
        TF\_IDF      & 0.812          & 0.885          & 0.835          & 0.907          & 0.835          & 0.905          & 0.824          & 0.906          & 0.764          & 0.867          & 0.773          & 0.853          & 0.807          & 0.887          \\

        \bottomrule
    \end{tabular}}
    \renewcommand{\arraystretch}{1.0}
\end{table}

% ===================
% Candidate Selection
% ===================
\section{Candidate Selection}\label{eval-candidate-selection}

\begin{table}[t]
    \centering
    \footnotesize
    % \caption{\texttt{Recall} for the retrieved candidates from the target corpus, along with the total number of candidates. The retrieval process allows multiple passages from the same document in the source dataset.}
    % \caption{Overview of the recall (Rec.) and the total number of candidate documents (Docs.) for all three candidate retrieval approaches. Each approach was tested with restricting the selection of documents to a maximum of one passage per document (opd.) and without (def.). Also, the nearest neighbor, and therefore the union appproach which combines the results of naive and nearest neighbor retrieval, were evaluated with different values of $k$ which specifies the number of used passages for candidate retrievel.}
    \caption{Overview of recall (Rec.) and the total number of candidate documents (Docs.) for all three candidate retrieval approaches, tested with document selection restricted to one passage per document (opd.) and without restriction (def.). The \texttt{Nearest Neighbor} and the \texttt{Union} approaches were also evaluated with varying $k$, determining the number of passages used for candidate retrieval.}
    \resizebox{\textwidth}{!}{%
    \renewcommand{\arraystretch}{1.2}
    \begin{tabular}{ccrcrcrcrcrcrcr}
        \toprule

        \multicolumn{3}{c}{\textbf{Approach}} & \multicolumn{2}{c}{\textbf{Touché 20}} & \multicolumn{2}{c}{\textbf{Robust04}} & \multicolumn{2}{c}{\textbf{TREC-7}} & \multicolumn{2}{c}{\textbf{TREC-8}} & \multicolumn{2}{c}{\textbf{TREC-19 DL}} & \multicolumn{2}{c}{\textbf{TREC-20 DL}} \\

        \cmidrule(lr){4-5} \cmidrule(lr){6-7} \cmidrule(lr){8-9} \cmidrule(lr){10-11} \cmidrule(lr){12-13} \cmidrule(lr){14-15}
                                          & & $k$ & Rec. & Docs. & Rec. & Docs. & Rec. & Docs. & Rec. & Docs. & Rec. & Docs. & Rec. & Docs. \\
        \midrule
        \multirow{2}{*}{\rotatebox{90}{\textbf{Naive}}} & def. & - & $0.912$ & $76\,284$ & $0.777$ &$ 391\,626$ & $0.657$ & $80\,032$ & $0.713$ & $78\,334$ & $0.708$ & $42\,006$ & $0.806$ & $44\,156$ \\
        \cmidrule(lr){2-15}
                                                        & opd. & - & $0.912$ & $76\,284$ & $0.777$ &$ 391\,626$ & $0.657$ & $80\,032$ & $0.713$ & $78\,334$ & $0.708$ & $42\,006$ & $0.806$ & $44\,156$ \\
        \midrule
        \multirow{6}{*}{\rotatebox{90}{\textbf{Nearest Neighbor}}} & \multirow{3}{*}{def.} &  10 & $0.599$ &  $5\,106$ & $0.520$ &  $22\,904$ & $0.446$ &  $4\,464$ & $0.454$ &  $4\,642$ & $0.419$ &  $3\,549$ & $0.583$ &  $4\,080$ \\
                                                                                        &  &  50 & $0.795$ & $13\,809$ & $0.726$ &  $84\,128$ & $0.707$ & $17\,573$ & $0.668$ & $17\,993$ & $0.595$ & $12\,043$ & $0.730$ & $13\,102$ \\
                                                                                        &  & 100 & $0.852$ & $16\,469$ & $0.790$ & $144\,213$ & $0.793$ & $32\,829$ & $0.750$ & $34\,089$ & $0.669$ & $20\,254$ & $0.775$ & $20\,134$ \\
       \cmidrule(lr){2-15}
                                                                   & \multirow{3}{*}{opd.}  &  10 & $0.707$ &  $5\,287$ & $0.577$ &  $24\,246$ & $0.508$ &  $5\,016$ & $0.495$ &  $5\,198$ & $0.444$ &  $3\,371$ & $0.618$ & $3\,405$ \\
                                                                                        &  &  50 & $0.816$ &  $7\,262$ & $0.809$ &  $79\,589$ & $0.829$ & $20\,818$ & $0.797$ & $22\,010$ & $0.624$ &  $8\,177$ & $0.712$ & $5\,884$ \\
                                                                                        &  & 100 & $0.816$ &  $7\,262$ & $0.813$ &  $81\,527$ & $0.829$ & $20\,818$ & $0.797$ & $22\,010$ & $0.658$ & $10\,322$ & $0.712$ & $5\,884$ \\
        \midrule
        \multirow{6}{*}{\rotatebox{90}{\textbf{Union}}} & \multirow{3}{*}{def.} &  10 & $0.932$ & $77\,509$ & $0.851$ & $403\,659$ & $0.769$ &  $82\,690$ & $0.797$ &  $80\,705$ & $0.771$ & $43\,616$ & $0.869$ & $46\,282$ \\
                                                                             &  &  50 & $0.964$ & $81\,753$ & $0.906$ & $451\,335$ & $0.862$ &  $93\,116$ & $0.864$ &  $90\,979$ & $0.826$ & $49\,939$ & $0.910$ & $53\,547$ \\
                                                                             &  & 100 & $0.982$ & $83\,473$ & $0.930$ & $503\,548$ & $0.905$ & $106\,555$ & $0.893$ & $105\,038$ & $0.859$ & $56\,783$ & $0.923$ & $59\,629$ \\
        \cmidrule(lr){2-15}
                                                        & \multirow{3}{*}{opd.}  &  10 & $0.949$ & $77\,401$ & $0.863$ & $404\,321$ & $0.792$ & $82\,938$ & $0.809$ & $81\,094$ & $0.780$ & $43\,398$ & $0.889$ & $45\,680$ \\
                                                                             &  &  50 & $0.982$ & $78\,136$ & $0.942$ & $447\,708$ & $0.916$ & $95\,672$ & $0.918$ & $94\,441$ & $0.843$ & $46\,290$ & $0.917$ & $47\,474$ \\
                                                                             &  & 100 & $0.982$ & $78\,136$ & $0.943$ & $449\,425$ & $0.916$ & $95\,672$ & $0.918$ & $94\,441$ & $0.862$ & $47\,961$ & $0.917$ & $47\,474$ \\

        \bottomrule 
    \end{tabular}}
    \renewcommand{\arraystretch}{1.0}
\end{table}


% ====================
% Pairwise Preferences
% ====================
\section{Pairwise Preferences}\label{eval-pairwise-preferences}

The first evaluation of the pairwise preferences involves inferring relevance scores from the original relevance judgments within the source dataset itself. This allows for an automatic evaluation using rank correlation, as the retrieval tasks provide predefined relevance judgments for comparison. Each dataset can be evaluated individually to verify whether the inferred scores align with the original relevance judgments. The second evaluation assesses the transfer of inferred relevance scores from the retrieval tasks to the target corpus, \texttt{ClueWeb22/b}. Since no existing relevance judgments are available for \texttt{ClueWeb22/b}, this evaluation cannot be automated. Instead, randomly selected documents for which relevance scores have been inferred are chosen for each query and manually reviewed. Human assessors then determine whether the inferred relevance scores accurately reflect the relevance of the corresponding documents to the queries.

\begin{table}[t]
    \centering
    \footnotesize
    \caption{Correlations of the inferred relevance judgements to the original document judgments reported in terms of Kendall's $\tau$ and Spearman's $\rho$.}
    \resizebox{\textwidth}{!}{%
    \renewcommand{\arraystretch}{1.0}
    \begin{tabular}{cclccccccccccccc}
        \toprule

        \multicolumn{3}{c}{\textbf{Approach}} & \multicolumn{2}{c}{\textbf{Touché 20}} & \multicolumn{2}{c}{\textbf{Robust04}} & \multicolumn{2}{c}{\textbf{TREC-7}} & \multicolumn{2}{c}{\textbf{TREC-8}} & \multicolumn{2}{c}{\textbf{TREC-19 DL}} & \multicolumn{2}{c}{\textbf{TREC-20 DL}} \\
        \cmidrule(lr){4-5} \cmidrule(lr){6-7} \cmidrule(lr){8-9} \cmidrule(lr){10-11} \cmidrule(lr){12-13} \cmidrule(lr){14-15}
                            & & & $\tau$ & $\rho$ & $\tau$ & $\rho$ & $\tau$ & $\rho$ & $\tau$ & $\rho$ & $\tau$ & $\rho$ & $\tau$ & $\rho$ \\
        \midrule
        \multirow{16}{*}{\rotatebox{90}{\makecell{\textbf{Pairwise Preferences} \\ \textbf{flant5-base}}}}
            & \multirow{4}{*}{\textbf{mean}} 
              & \textbf{id}   & 0.000 & 0.000 & 0.000 & 0.000 & 0.000 & 0.000 & 0.000 & 0.000 & 0.000 & 0.000 & 0.000 & 0.000 \\
            & & \textbf{log}  & 0.000 & 0.000 & 0.000 & 0.000 & 0.000 & 0.000 & 0.000 & 0.000 & 0.000 & 0.000 & 0.000 & 0.000 \\
            & & \textbf{exp}  & 0.000 & 0.000 & 0.000 & 0.000 & 0.000 & 0.000 & 0.000 & 0.000 & 0.000 & 0.000 & 0.000 & 0.000 \\
            & & \textbf{sqrt} & 0.000 & 0.000 & 0.000 & 0.000 & 0.000 & 0.000 & 0.000 & 0.000 & 0.000 & 0.000 & 0.000 & 0.000 \\
        \cmidrule{2-15}
            & \multirow{4}{*}{\textbf{min}} 
              & \textbf{id}   & 0.000 & 0.000 & 0.000 & 0.000 & 0.000 & 0.000 & 0.000 & 0.000 & 0.000 & 0.000 & 0.000 & 0.000 \\
            & & \textbf{log}  & 0.000 & 0.000 & 0.000 & 0.000 & 0.000 & 0.000 & 0.000 & 0.000 & 0.000 & 0.000 & 0.000 & 0.000 \\
            & & \textbf{exp}  & 0.000 & 0.000 & 0.000 & 0.000 & 0.000 & 0.000 & 0.000 & 0.000 & 0.000 & 0.000 & 0.000 & 0.000 \\
            & & \textbf{sqrt} & 0.000 & 0.000 & 0.000 & 0.000 & 0.000 & 0.000 & 0.000 & 0.000 & 0.000 & 0.000 & 0.000 & 0.000 \\
        \cmidrule{2-15}
            & \multirow{4}{*}{\textbf{max}} 
              & \textbf{id}   & 0.000 & 0.000 & 0.000 & 0.000 & 0.000 & 0.000 & 0.000 & 0.000 & 0.000 & 0.000 & 0.000 & 0.000 \\
            & & \textbf{log}  & 0.000 & 0.000 & 0.000 & 0.000 & 0.000 & 0.000 & 0.000 & 0.000 & 0.000 & 0.000 & 0.000 & 0.000 \\
            & & \textbf{exp}  & 0.000 & 0.000 & 0.000 & 0.000 & 0.000 & 0.000 & 0.000 & 0.000 & 0.000 & 0.000 & 0.000 & 0.000 \\
            & & \textbf{sqrt} & 0.000 & 0.000 & 0.000 & 0.000 & 0.000 & 0.000 & 0.000 & 0.000 & 0.000 & 0.000 & 0.000 & 0.000 \\
        \cmidrule{2-15}
            & \multirow{4}{*}{\textbf{sum}} 
              & \textbf{id}   & 0.000 & 0.000 & 0.000 & 0.000 & 0.000 & 0.000 & 0.000 & 0.000 & 0.000 & 0.000 & 0.000 & 0.000 \\
            & & \textbf{log}  & 0.000 & 0.000 & 0.000 & 0.000 & 0.000 & 0.000 & 0.000 & 0.000 & 0.000 & 0.000 & 0.000 & 0.000 \\
            & & \textbf{exp}  & 0.000 & 0.000 & 0.000 & 0.000 & 0.000 & 0.000 & 0.000 & 0.000 & 0.000 & 0.000 & 0.000 & 0.000 \\
            & & \textbf{sqrt} & 0.000 & 0.000 & 0.000 & 0.000 & 0.000 & 0.000 & 0.000 & 0.000 & 0.000 & 0.000 & 0.000 & 0.000 \\
        \midrule

        \multirow{16}{*}{\rotatebox{90}{\makecell{\textbf{Pairwise Preferences} \\ \textbf{flant5-small}}}}
            & \multirow{4}{*}{\textbf{mean}} 
              & \textbf{id}   & 0.000 & 0.000 & 0.000 & 0.000 & 0.000 & 0.000 & 0.000 & 0.000 & 0.000 & 0.000 & 0.000 & 0.000 \\
            & & \textbf{log}  & 0.000 & 0.000 & 0.000 & 0.000 & 0.000 & 0.000 & 0.000 & 0.000 & 0.000 & 0.000 & 0.000 & 0.000 \\
            & & \textbf{exp}  & 0.000 & 0.000 & 0.000 & 0.000 & 0.000 & 0.000 & 0.000 & 0.000 & 0.000 & 0.000 & 0.000 & 0.000 \\
            & & \textbf{sqrt} & 0.000 & 0.000 & 0.000 & 0.000 & 0.000 & 0.000 & 0.000 & 0.000 & 0.000 & 0.000 & 0.000 & 0.000 \\
        \cmidrule{2-15}
            & \multirow{4}{*}{\textbf{min}} 
              & \textbf{id}   & 0.000 & 0.000 & 0.000 & 0.000 & 0.000 & 0.000 & 0.000 & 0.000 & 0.000 & 0.000 & 0.000 & 0.000 \\
            & & \textbf{log}  & 0.000 & 0.000 & 0.000 & 0.000 & 0.000 & 0.000 & 0.000 & 0.000 & 0.000 & 0.000 & 0.000 & 0.000 \\
            & & \textbf{exp}  & 0.000 & 0.000 & 0.000 & 0.000 & 0.000 & 0.000 & 0.000 & 0.000 & 0.000 & 0.000 & 0.000 & 0.000 \\
            & & \textbf{sqrt} & 0.000 & 0.000 & 0.000 & 0.000 & 0.000 & 0.000 & 0.000 & 0.000 & 0.000 & 0.000 & 0.000 & 0.000 \\
        \cmidrule{2-15}
            & \multirow{4}{*}{\textbf{max}} 
              & \textbf{id}   & 0.000 & 0.000 & 0.000 & 0.000 & 0.000 & 0.000 & 0.000 & 0.000 & 0.000 & 0.000 & 0.000 & 0.000 \\
            & & \textbf{log}  & 0.000 & 0.000 & 0.000 & 0.000 & 0.000 & 0.000 & 0.000 & 0.000 & 0.000 & 0.000 & 0.000 & 0.000 \\
            & & \textbf{exp}  & 0.000 & 0.000 & 0.000 & 0.000 & 0.000 & 0.000 & 0.000 & 0.000 & 0.000 & 0.000 & 0.000 & 0.000 \\
            & & \textbf{sqrt} & 0.000 & 0.000 & 0.000 & 0.000 & 0.000 & 0.000 & 0.000 & 0.000 & 0.000 & 0.000 & 0.000 & 0.000 \\
        \cmidrule{2-15}
            & \multirow{4}{*}{\textbf{sum}} 
              & \textbf{id}   & 0.000 & 0.000 & 0.000 & 0.000 & 0.000 & 0.000 & 0.000 & 0.000 & 0.000 & 0.000 & 0.000 & 0.000 \\
            & & \textbf{log}  & 0.000 & 0.000 & 0.000 & 0.000 & 0.000 & 0.000 & 0.000 & 0.000 & 0.000 & 0.000 & 0.000 & 0.000 \\
            & & \textbf{exp}  & 0.000 & 0.000 & 0.000 & 0.000 & 0.000 & 0.000 & 0.000 & 0.000 & 0.000 & 0.000 & 0.000 & 0.000 \\
            & & \textbf{sqrt} & 0.000 & 0.000 & 0.000 & 0.000 & 0.000 & 0.000 & 0.000 & 0.000 & 0.000 & 0.000 & 0.000 & 0.000 \\
        \midrule

        \multirow{16}{*}{\rotatebox{90}{\makecell{\textbf{Pairwise Preferences} \\ \textbf{t5-small}}}}
            & \multirow{4}{*}{\textbf{mean}} 
              & \textbf{id}   & 0.000 & 0.000 & 0.000 & 0.000 & 0.000 & 0.000 & 0.000 & 0.000 & 0.000 & 0.000 & 0.000 & 0.000 \\
            & & \textbf{log}  & 0.000 & 0.000 & 0.000 & 0.000 & 0.000 & 0.000 & 0.000 & 0.000 & 0.000 & 0.000 & 0.000 & 0.000 \\
            & & \textbf{exp}  & 0.000 & 0.000 & 0.000 & 0.000 & 0.000 & 0.000 & 0.000 & 0.000 & 0.000 & 0.000 & 0.000 & 0.000 \\
            & & \textbf{sqrt} & 0.000 & 0.000 & 0.000 & 0.000 & 0.000 & 0.000 & 0.000 & 0.000 & 0.000 & 0.000 & 0.000 & 0.000 \\
        \cmidrule{2-15}
            & \multirow{4}{*}{\textbf{min}} 
              & \textbf{id}   & 0.000 & 0.000 & 0.000 & 0.000 & 0.000 & 0.000 & 0.000 & 0.000 & 0.000 & 0.000 & 0.000 & 0.000 \\
            & & \textbf{log}  & 0.000 & 0.000 & 0.000 & 0.000 & 0.000 & 0.000 & 0.000 & 0.000 & 0.000 & 0.000 & 0.000 & 0.000 \\
            & & \textbf{exp}  & 0.000 & 0.000 & 0.000 & 0.000 & 0.000 & 0.000 & 0.000 & 0.000 & 0.000 & 0.000 & 0.000 & 0.000 \\
            & & \textbf{sqrt} & 0.000 & 0.000 & 0.000 & 0.000 & 0.000 & 0.000 & 0.000 & 0.000 & 0.000 & 0.000 & 0.000 & 0.000 \\
        \cmidrule{2-15}
            & \multirow{4}{*}{\textbf{max}} 
              & \textbf{id}   & 0.000 & 0.000 & 0.000 & 0.000 & 0.000 & 0.000 & 0.000 & 0.000 & 0.000 & 0.000 & 0.000 & 0.000 \\
            & & \textbf{log}  & 0.000 & 0.000 & 0.000 & 0.000 & 0.000 & 0.000 & 0.000 & 0.000 & 0.000 & 0.000 & 0.000 & 0.000 \\
            & & \textbf{exp}  & 0.000 & 0.000 & 0.000 & 0.000 & 0.000 & 0.000 & 0.000 & 0.000 & 0.000 & 0.000 & 0.000 & 0.000 \\
            & & \textbf{sqrt} & 0.000 & 0.000 & 0.000 & 0.000 & 0.000 & 0.000 & 0.000 & 0.000 & 0.000 & 0.000 & 0.000 & 0.000 \\
        \cmidrule{2-15}
            & \multirow{4}{*}{\textbf{sum}} 
              & \textbf{id}   & 0.000 & 0.000 & 0.000 & 0.000 & 0.000 & 0.000 & 0.000 & 0.000 & 0.000 & 0.000 & 0.000 & 0.000 \\
            & & \textbf{log}  & 0.000 & 0.000 & 0.000 & 0.000 & 0.000 & 0.000 & 0.000 & 0.000 & 0.000 & 0.000 & 0.000 & 0.000 \\
            & & \textbf{exp}  & 0.000 & 0.000 & 0.000 & 0.000 & 0.000 & 0.000 & 0.000 & 0.000 & 0.000 & 0.000 & 0.000 & 0.000 \\
            & & \textbf{sqrt} & 0.000 & 0.000 & 0.000 & 0.000 & 0.000 & 0.000 & 0.000 & 0.000 & 0.000 & 0.000 & 0.000 & 0.000 \\
        \midrule

        \multirow{5}{*}{\rotatebox{90}{\makecell{\textbf{Pointwise} \\ \textbf{Preferences}}}}
            & & & & & & & & & & & & & & \\
            & \multicolumn{2}{c}{\textbf{flant5-base}}   & 0.000 & 0.000 & 0.000 & 0.000 & 0.000 & 0.000 & 0.000 & 0.000 & 0.000 & 0.000 & 0.000 & 0.000 \\
            & \multicolumn{2}{c}{\textbf{flant5-small}}   & 0.000 & 0.000 & 0.000 & 0.000 & 0.000 & 0.000 & 0.000 & 0.000 & 0.000 & 0.000 & 0.000 & 0.000 \\
            & \multicolumn{2}{c}{\textbf{t5-small}}   & 0.000 & 0.000 & 0.000 & 0.000 & 0.000 & 0.000 & 0.000 & 0.000 & 0.000 & 0.000 & 0.000 & 0.000 \\
            & & & & & & & & & & & & & & \\
        \bottomrule 
    \end{tabular}}
    \renewcommand{\arraystretch}{1.0}
\end{table}

% Transfer Pipeline to Source Corpus itself
\subsection{Transfer Pipeline to Source Corpus itself}\label{eval-pairwise-preferences-source}

% Transfer Pipeline to ClueWeb22/b
\subsection{Transfer Pipeline to \texttt{ClueWeb22/b}}\label{eval-pairwise-preferences-target}