\chapter{Conclusion \& Future Work}\label{conclusion}

This thesis presented an approach to automatically generate relevance judgments using an existing annotated test collection. This collection consisted of a document corpus and a retrieval task with predefined queries and relevance judgments. To enable the transfer of relevance judgments to another document corpus, a transfer pipeline based on pairwise comparisons was developed. The evaluation of the pipeline showed that the pairwise preference approach significantly outperforms a pointwise approach, both in a self-transfer setting and in the generation of relevance judgments for \texttt{ClueWeb22/b}. This indicates that comparing an unjudged document with already judged documents improves the quality of automatically generated judgments. However, successful transfer requires a source dataset with a sufficient number of high-quality relevance judgments. While the pairwise preference approach shows great potential for reducing the manual effort required to generate relevance judgments, it does not yet achieve the same quality as human judgments. Therefore, manual judgments are still necessary, and automatic generation is currently best suited for pre-judgment tasks rather than final relevance judgments.
% In this thesis, an approach for automatically generating relevance judgments by using an existing annotated test collection comprising a document corpus and a retrieval task with predefined queries and relevance judgments was presented. To facilitate the transfer of relevance judgments to a different document corpus, a transfer pipeline that automates this process was developed, by comparing unjudged documents to already judged documents for the same query. The evaluation of the pipeline, both in a self-transfer setting and for generating relevance judgments for \texttt{ClueWeb22/b}, revealed that the pairwise preference approach significantly outperformed the pointwise approach. These results demonstrate that the pairwise approach enhances the quality of automatically generated relevance judgments. However, successful transfer relies on a source dataset containing a sufficient number of high-quality relevance judgments.While the pairwise preference approach shows great potential in reducing the manual effort needed to create relevance judgments, it does not yet achieve the same quality as human assessments. Therefore, manual judgments remain necessary, and automatic relevance generation is currently best suited for pre-judgment tasks rather than final relevance assessments.
\\\\
The comparison of different \texttt{T5} model variants showed that more fine-tuned models achieved higher rank correlations. It is therefore likely that a larger version of \texttt{FLAN-T5} with more parameters would yield even better relevance predictions. Future work could explore alternative large language models and improved prompting strategies, such as incorporating a query's description or narrative, for pairwise preference inference. Another key component of the transfer pipeline was the \texttt{Union opd.\ 100} approach, which proved to be the most effective candidate selection method. However, refining candidate selection remains an open research task. The nearest-neighbor approach already showed strong performance, and its combination with the naive selection method further increased recall, albeit at the cost of a larger candidate set.
