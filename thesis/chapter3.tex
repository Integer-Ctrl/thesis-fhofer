\chapter{Methodology}\label{methodology}

In this Chapter, I present the individual steps of the pipeline used to process the datasets. I begin by introducing all the datasets on which the transfer process was performed and evaluated. Subsequently, I discuss each step of the processing pipeline in detail. The goal is to automatically infer \texttt{qrels}for the target dataset through pairwise preferences based on the existing \texttt{qrels} of the source dataset.

The first step in the processing pipeline involves segmenting a selection of documents in the source dataset, that contain at least one \texttt{qrel}, into passages. Relevant information within a document, if present, is typically confined to specific parts and not in the whole document. To improve the results in the pairwise preferences later, the goal is on filtering out only those passages that are highly relevant to a query.

The second step identifies the passages of a document that are most relevant to the query of its \texttt{qrel}. For each \texttt{qrel} in the source dataset, i.e., the \texttt{(query, document, label)} triples, all passages of the document are treated as individual queries and submitted as requests against the source dataset. Based on the responses to these requests, various metrics are calculated to measure the proportion of relevant documents retrieved in the document ranking. Passages containing a large amount of relevant information for the query of the \texttt{qrel} retrieve more relevant documents and achieve better metrics than those contributing little or no information to the query.

In the third step, the quality of the assigned passage scores is evaluated using different rank correlation methods. Based on the evaluation results, the metric with the highest rank correlation is used to select candidates for the pairwise preferences. In this selection phase, documents, referred to as candidates, are identified in the target dataset for which \texttt{qrels} will be determined. A candidate, similar to a \texttt{qrel}, consists of a query $q$, a candidate $c$, and a set of known, relevant passages for $q$ from the source dataset. Since the relevant texts used later for pairwise preference are passages rather than documents, each candidate $c$ is also divided into passages $c=(c_1, c_2, ..., c_n)$.

In the final step, the pairwise preferences are inferred. All candidates, i.e., triples consisting of \texttt{(query text, relevant passage text, unknown passage text)}, are fed into a pairwise ranking model to determine whether the unknown passage is as relevant as the relevant passage concerning the query text. From the results of the pairwise preferences, labels for the respective unknown passages are generated through the aggregation of all pairwise preferences with the $k$ top relevant passages. A label for an entire document in the target Dataset is then created by aggregating the labels of all its passages.

\section{Datasets}\label{datasets}

\begin{table}[h!]
    \centering
    \caption{List of source datasets and their associated retrieval tasks from which existing \texttt{qrels} where transferd into the target dataset \texttt{ClueWeb22/b}.}
    \label{table:datasets}
    \begin{tabular}{crcrrr}
        \toprule
        \multicolumn{2}{c}{\textbf{Corpus}} & \multicolumn{4}{c}{\textbf{ Associated Retrieval Tasks}} \\
        \cmidrule(lr){1-2} \cmidrule(lr){3-6}
        Name & Documents  & Name & Queries & Qrels & Labels \\
        \toprule
        
        Args.me & 0.4~m & Touché 2020 Task 1 & 49 & 2,298 & 3 \\
        \midrule

        \multirow{4}{*}{ClueWeb09} & \multirow{4}{*}{1.0~b} & TREC 2009 Web Track & 50 & 23,601 & 3 \\
        & & TREC 2010 Web Track & 50 & 25,329 & 4 \\
        & & TREC 2011 Web Track & 50 & 19,381 & 4\\
        & & TREC 2012 Web Track & 50 & 16,055 & 5 \\
        \midrule

        \multirow{4}{*}{ClueWeb12} & \multirow{4}{*}{731.7~m} & TREC 2013 Web Track & 50 & 14,474 & 5 \\
        & & TREC 2014 Web Track & 50 & 14,432 & 5 \\
        & & Touché 2021 Task 2 & 50 & 2,076 & 3 \\
        & & Touché 2022 Task 2 & 50 & 2,107 & 3 \\
        \midrule

        \multirow{3}{*}{Disks4+5} & \multirow{3}{*}{0.5~m} & Robust04 & 250 & 311,410 & 3 \\
        & & TREC-7 & 50 & 80,345 & 2 \\
        & & TREC-8 & 50 & 86,830 & 2 \\
        \midrule

        \multirow{2}{*}{MS MARCO} & \multirow{2}{*}{3.2~m} & TREC 2019 DL Track & 43 & 16,258 & 4 \\
        & & TREC 2020 DL Track & 45 & 9,098 & 4 \\
        
        \bottomrule
    \end{tabular}
\end{table}

To simplify data handling, I used \texttt{ir\_datasets}~\citep{macavaney:2021}, a Python package that provides numerous datasets and their associated retrieval tasks. The advantage of \texttt{ir\_datasets} is that it provides a standardized interface for accessing data. Using various iterators, the package manages access to corpora, queries, and \texttt{qrels}, enabling the processing pipeline to handle different datasets in a uniform way. The datasets used in this research are listed in Table~\ref{table:datasets}. The source datasets were selected because they are widely used in the field of information retrieval and already contain relevance judgments. These relevance judgments from the source datasets are later transferd to the target dataset, \texttt{ClueWeb22/b}. The target dataset, \texttt{ClueWeb22/b}, was chosen because it is currently the newest ClueWeb corpus in the \texttt{Lemur Project}\footnote{https://lemurproject.org}. It is significantly large, containing over 1.0~billion documents. Due to its size, the ratio of \texttt{qrels} to documents is relatively low. Therefore, the goal is to leverage the existing \texttt{qrels} from the source datasets and their associated retrieval tasks to generate new \texttt{qrels} for ClueWeb22/b, thereby enriching the corpus with additional relevance judgments. 

\section{Documents Segmentation}\label{document-segmentation}



The first step in the research process was to convert the documents in the datasets to passages. This was done to reduce the size of the documents and to make the relevance judgements more fine-grained.

\begin{itemize}
    \item Why was this step necessary?
    \item How were the documents split into passages?
    \item Usage of \href{https://github.com/grill-lab/trec-cast-tools/tree/master/corpus_processing/passage_chunkers}{trec-cast-tools}
    \item Passage length
    \item Number of passages per document
\end{itemize}

\section{Passage Scoring}\label{passage-scoring}

The next step was to infer relevance scores for the passages. This was done by using the relevance judgements, qrels, of the documents in the original datasets. For each qrel in the dataset with a label of 1 or 2, the corresponding passages of the document were used as a query to retrieve the documents of the dataset. 

\begin{itemize}
    \item Why was this step done?
    \item How were the relevance scores inferred?
    \item Usage of \href{https://github.com/joaopalotti/trectools}{trectools}
    \item What is a qrel?
    \item Number of retrieved documents per query
    \item What metrics were used to evaluate the retrieval performance?
\end{itemize}

\section{Rank Correlation}\label{rank-correlation-scores}

The relevance scores of the passages were then used to assign relevance labels to the passages. To do this, I used to open source tool \href{https://github.com/seanmacavaney/autoqrels}{autoqrels}. The tool can be used to automatically assign relevance labels to passages based on the relevance scores of the passages.

\begin{itemize}
    \item Why was this step done?
    \item How were the relevance labels assigned?
    \item Usage of \href{https://github.com/seanmacavaney/autoqrels}{autoqrels}
    \item What is a relevance label?
\end{itemize}


\section{Pairwise Preferences}\label{pairwise-transfering-relevance-labels-across-datasets}

To transfer relevance labels from the old dataset to the new dataset, the DuoT5 transformer model was utilized. This model takes a query and two documents as input and outputs a relevance score, which represents the probability that the first document is more relevant than the second. Here, a "document" refers to a passage. To assign a relevance label to a passage in the new dataset, it is compared against the top 20 to 30 passages for the same query in the old dataset. Pairwise comparison results and their associated queries are cached to avoid redundant calculations, and the final relevance score is determined by averaging the pairwise results. Two approaches are used to select passages for comparison, as detailed below.

\subsection{Pairwise Preferences Approach 1}\label{pairwise-preferences-approach-1}

This approach identifies 20 to 30 passages from the old dataset most relevant to the query for which the new passage is being labeled. To prevent biases from comparing passages within the same document, all passages from the same document as the first passage are excluded. The first passage is the highest-scoring passage from its document, the second passage is the next highest-scoring passage from another document, and so on. The relevance score for the new passage is then calculated by averaging the scores from pairwise comparisons with the selected top passages.

\subsection{Pairwise Preferences Approach 2}\label{pairwise-preferences-approach-2}

This approach is similar to the first but employs an additional step to eliminate overlaps with the same document. After selecting the highest-scoring passage for a query, the retrieval scores for all passages are recomputed, excluding the document of the already-selected passage from the retrieved documents. This ensures no passages from the same document are used more than once in the comparison. As with the first approach, the final relevance score is obtained by averaging the pairwise comparison scores between the new passage and the top 20 to 30 passages from the old dataset.