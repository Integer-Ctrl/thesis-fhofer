\chapter{Related Work}\label{related-work}

\cite{froebe:2021} addresses the issue of near-duplicate documents in widely-used web crawls, such as ClueWeb09, ClueWeb12, and Common Crawl, which can have a distorting effect on the reliability and validity of information retrieval experiments. The paper proposes the possibility of transferring relevance judgments from older crawls to newer ones, which is the area I aim to investigate further in this thesis. While the authors found that some relevance judgments could be reused, they concluded that new judgments are still necessary for shared tasks on newer datasets. This research will build on this work by focusing on improving the effectiveness of transferring these judgments across different crawls.
\\\\
The objective of the \citet{gienapp:2022} project was to enhance the efficiency of pairwise re-ranking models in information retrieval while maintaining effectiveness. However, these models typically face challenges due to high inference overhead, which is caused by the quadratic number of document comparisons required. To overcome this issue, three sampling methods and five preference aggregation methods were evaluated. The results demonstrated that re-ranking effectiveness could be achieved with only one-third of the usual comparisons, and even fewer comparisons could yield only a slight decrease in effectiveness.
\\\\
In the field of information retrieval, \citet{mackenzie:2021} evaluated the impact of considering multiple relevant passages per query on the performance assessment of retrieval systems using the MSMARCO passage dataset. The objective was to ascertain the impact of the addition of plausible but previously unjudged relevant passages on system scores and comparative rankings. The results demonstrated that the addition of plausible, relevant passages can result in considerable variation in individual run scores, while the overall system rankings remain relatively stable. These results provide support for the methodologies used in constructing the MSMARCO passage collection.
\\\\
The research conducted by pradeep2021 investigates a multi-stage ranking approach for text retrieval tasks. The "Expando" component uses document expansion to improve keyword representation before retrieval. The "Mono" component applies a pointwise ranking model, and the "Duo" component employs a pairwise model for final reranking. This design pattern, validated on multiple datasets like MS MARCO and TREC-COVID, achieves near state-of-the-art performance. The proposed architecture improves ranking effectiveness with an additive benefit from all components and can even work in zero-shot scenarios. This research will build on these findings by investigating the effectiveness of using large language models for transferring relevance judgments across different datasets.
\\\\
\citet{pradeep:2021} examined the potential of a multi-stage ranking approach for text retrieval. The \"Expando" component employed document expansion using a sequence-to-sequence model to enhance the representation of keywords in texts. The "Mono" component applied a pointwise ranking model, while the "Duo" component utilised a pairwise model for final reranking. The design was validated on datasets such as MS MARCO and TREC-COVID, achieving near state-of-the-art performance and demonstrating effectiveness even in zero-shot scenarios. Building on these findings, the present research will examine the use of large language models to transfer relevance judgments across different datasets.
\\\\
The most commonly employed methods for zero-shot ranking of documents utilising large language models are Pointwise, Pairwise and Listwise. The research conducted by \citet{zhuang:2024} proposed a new approach to zero-shot document ranking, termed "Setwise prompting". The objective of this method is to enhance the existing approaches by reducing the number of inferences during ranking, thereby achieving a more balanced trade-off between efficiency and effectiveness. The study presents a comprehensive and systematic comparison of these methods, thereby demonstrating the advantages of Setwise in the context of zero-shot ranking.
\\\\
\cite{macavaney:2023} examines the potential of large language models (LLMs) to fill "holes" in relevance assessments for information retrieval evaluations, particularly in cases where only one relevant document per query is available. It proposes One-Shot Labelers ($\mathbb{1}$SL) that utilize techniques like nearest neighbor searches, supervised models, and prompting to predict relevance for unjudged documents. While these methods have been shown to have limitations in identifying all relevant documents, they achieve a high correlation with rankings from full assessments using recall-agnostic measures.The study suggests that $\mathbb{1}$SL can complement human annotations, improving less time and expanding evaluation scope, though not fully replacing manual relevance judgments. The instruction-tuned model FlanT5 (\citet{chung:2022}) turned out to be a very reliable method to infer relevance labels for the unjudged documents. Therefore, this model is also used in this work to take a query, a relevant document for this query and a document that is not judged. The model infers a value for the unjudged document in terms of relevance to the query.
