\chapter{Related Work}\label{related-work}

The idea of transferring relevance judgments from one dataset to another has been explored in multiple studies. \citet{froebe:2021} investigated the issue of near-duplicate documents within widely-used web crawls such as \texttt{ClueWeb} and \texttt{Common Crawl}. The authors proposed a deduplication approach to address this issue and, as an extension, explored the potential of transferring relevance judgments between near-duplicate documents. However, their study also emphasized that newly collected judgments remain necessary for evaluating retrieval systems on newer datasets. An example of ineffective relevance judgment transfer was conducted on the  \texttt{MS MARCO} corpus. The \mbox{\texttt{MS MARCO} crawl} is a widely used training and test dataset in information retrieval, available both as a document corpus and as a passage corpus. \citet{froebe:2022} analyzed  why retrieval models trained on \texttt{MS MARCO v1} performed better than those trained on \texttt{MS MARCO  v2}. In version one, the passage dataset was available and judged before the document dataset. To take advantage of this, relevance judgments were directly transferred from the passage dataset to the document dataset by assigning the same relevance judgment to any document that had the same URL as the judged passage. A similar process was applied to enrich version two, where documents were assigned the same relevance judgment as their corresponding documents from version one if they had the same URL. The issue with this approach was that the document corpus in version one was crawled one year after the passage corpus, during which time the content of URLs remained largely unchanged. However, the document corpus in version two was crawled four years after the passage corpus, leading to many content changes. This dissimilarity resulted in inaccurate relevance transfers. This example highlights the importance of considering document content when transferring relevance judgments between datasets. To address this challenge, the transfer pipeline in this thesis employs a pairwise preference approach to compare old and new documents, ensuring a more reliable transfer of relevance information.
\\\\
A more automated approach to infer relevance judgments was introduced by \citet{macavaney:2023}, who proposed One-Shot Labelers ($\mathbb{1}$SL) to predict relevance for unjudged passages using nearest neighbor searches and various prompting strategies. Their goal was to examine the potential of large language models to fill \glqq holes\grqq{} (i.e., unjudged documents) in relevance assessments for information retrieval evaluations. Their results demonstrated that instruction-tuned models, such as \texttt{FLAN-T5} \citep{chung:2022}, can provide reliable relevance estimations. However, their study focused only on passages and left document level inference as future work. This thesis addresses that gap by employing a pairwise inference approach with \texttt{FLAN-T5} to generate \mbox{labels at document level}.
\\\\
Re-ranking methods play a crucial role in improving document ranking effectiveness. Traditional approaches include pointwise, pairwise, and listwise ranking, each with a trade-off between efficiency and effectiveness. \citet{pradeep:2021} introduced a multi-stage ranking framework that combines document expansion, pointwise ranking, and pairwise re-ranking to enhance retrieval performance. Their findings showed that pairwise ranking models, such as \texttt{DuoT5}, significantly improve ranking quality, even in zero-shot scenarios. Inspired by this, the relevance transfer pipeline in this thesis leverages pairwise inference to determine the relevance of documents in the target corpus.
\\\\
One of the key challenges in the relevance transfer pipeline is the selection of candidate documents from the target corpus for relevance transfer. The selection is a critical step, as it determines which documents should be considered for relevance inference. \citet{macavaney:2022} highlighted a major limitation in re-ranking pipelines. They depend on the recall of the initial candidate pool, meaning that documents not retrieved in this initial stage cannot be re-ranked later. To address this issue, they proposed a graph-based adaptive re-ranking approach, which expands the candidate pool beyond the initial retrieval set by leveraging the clustering hypothesis \mbox{\citep{jardine:1971}}. This hypothesis states that closely related documents are often relevant to the same queries. Following this idea, this thesis explores various nearest neighbor strategies for selecting documents for relevance transfer.
\\\\\\\\\\\\\\
Processing pairwise preferences can be computationally expensive, particularly when dealing with large document collections, since it requires evaluating $k^2-k$ preferences for $k$ documents. To save computing resources, \mbox{\citet{gienapp:2022}} investigated whether sampling from all possible preference pairs could improve the efficiency of pairwise re-ranking models without sacrificing effectiveness. Their findings showed that re-ranking effectiveness could be maintained with only one-third of the usual comparisons, with only a minor performance drop when further reducing comparisons. Based on that, the number of pairwise comparisons in the relevance transfer pipeline must not be exhaustive to maintain effective. Similarly, \citet{zhuang:2024} introduced \glqq Setwise prompting \grqq{}, a novel zero-shot document ranking approach designed to reduce the number of required inferences while balancing efficiency and effectiveness. Their results demonstrated the advantages of Setwise prompting over traditional pairwise methods, suggesting that it could serve as a more fine-grained alternative for pairwise inference in relevance transfer.
\\\\
Overall, these studies provide essential insights into relevance transfer, candidate retrieval, and efficiency optimizations, forming the foundation for the relevance transfer pipeline developed in this thesis.